\documentclass[11pt]{article}
\usepackage[margin=1in]{geometry}
\usepackage{amsmath,amssymb,bm}
\usepackage{physics}
\usepackage{mathtools}
% \usepackage{hyperref}
% \hypersetup{
%     colorlinks=true,
%     linkcolor=blue,
%     filecolor=cyan,
%     urlcolor=magenta,
%     }

\newcommand{\ii}{\mathrm{i}}
\newcommand{\eps}{\varepsilon}

\begin{document}

\title{Particular case: Isotropic $\chi^{(3)}$ for counter- and co-rotating pumps}

\maketitle

\tableofcontents

\section{Isotropic \(\chi^{(3)}\) with Counter-Rotating Pumps:
Probe, Sidebands, and Cascaded Back-Action in Tensor Notation}

\paragraph{Fields and basis.}
Time convention \(e^{-\ii\omega t}\).
Let \( \{e^+_i,e^-_i,e^z_i\} \) be the circular basis with
\(e^+_i e^+_i=0\), \(e^-_i e^-_i=0\), \(e^+_i e^-_i=1\), and \((e^+_i)^*=e^-_i\).
We take
\[
E^{(1)}_i=E_1 e^+_i,\qquad
E^{(2)}_i=E_2 e^-_i,\qquad
E^{(s)}_i=E_+ e^+_i+E_- e^-_i .
\]
Define \(\Omega_\pm:=\omega_s\pm\Delta\) with \(\Delta=\omega_1-\omega_2\).

\paragraph{Isotropic \(\chi^{(3)}\) decomposition.}
For any frequency tuple \(\Xi\) (explicitly indicated below),
\[
\chi^{(3)}_{ijkl}(\Xi)=A(\Xi)\,\delta_{ij}\delta_{kl}
\;+\;B(\Xi)\,\delta_{ik}\delta_{jl}
\;+\;C(\Xi)\,\delta_{il}\delta_{jk}.
\]
We will keep \(\,A,B,C\,\) \emph{distinct for each tuple}.

\subsection{Direct third-order polarization at \(\omega_s\)}
Keeping only terms linear in \(E^{(s)}\) (SPM on the probe and XPM from the pumps),
\begin{align}
P^{(3)}_i(\omega_s)
=\frac{3}{4}\eps_0\sum_{m\in\{s,1,2\}}
\Big[&A_s^{(m)}\,\delta_{ij}\delta_{kl} + B_s^{(m)}\,\delta_{ik}\delta_{jl}
+ C_s^{(m)}\,\delta_{il}\delta_{jk}\Big]\,
E^{(s)}_j E^{(m)}_k E^{(m)*}_l,
\label{eq:P3-omegaS-general}
\end{align}
with shorthand
\[
A_s^{(m)}:=A\!\big(-\omega_s;\omega_s,\omega_m,-\omega_m\big),\;\;\;
B_s^{(m)}:=B\!\big(-\omega_s;\omega_s,\omega_m,-\omega_m\big),\;\;\;
C_s^{(m)}:=C\!\big(-\omega_s;\omega_s,\omega_m,-\omega_m\big).
\]
Insert the pump/probe polarizations. Using
\(
|E^{(1)}|^2=|E_1|^2,\ |E^{(2)}|^2=|E_2|^2,\ 
(E^{(s)}\!\cdot\!E^{(1)})=E_-E_1,\
(E^{(s)}\!\cdot\!E^{(1)*})=E_+E_1^*,\
(E^{(s)}\!\cdot\!E^{(2)})=E_+E_2,\
(E^{(s)}\!\cdot\!E^{(2)*})=E_-E_2^*,
\)
and \(E^{(s)}\!\cdot\!E^{(s)}=2E_+E_-\), one finds
\[
P^{(3)}_i(\omega_s)=P_+\,e^+_i+P_-\,e^-_i,
\]
with scalar envelopes
\begin{align}
P_+&=\frac{3}{4}\eps_0\Big[
(A_s^{(s)}+C_s^{(s)})\big(|E_+|^2+|E_-|^2\big)\,E_+
+ 2 B_s^{(s)}|E_-|^2 E_+
\nonumber\\[-2pt]
&\hspace{3.1em}
+ A_s^{(1)}|E_1|^2E_+ + A_s^{(2)}|E_2|^2E_+
+ B_s^{(1)}|E_1|^2 E_+ + C_s^{(2)}|E_2|^2 E_+
\Big],\label{eq:Pplus-direct}
\\[2pt]
P_-&=\frac{3}{4}\eps_0\Big[
(A_s^{(s)}+C_s^{(s)})\big(|E_+|^2+|E_-|^2\big)\,E_-
+ 2 B_s^{(s)}|E_+|^2 E_-
\nonumber\\[-2pt]
&\hspace{3.1em}
+ A_s^{(2)}|E_2|^2E_- + A_s^{(1)}|E_1|^2E_-
+ C_s^{(1)}|E_1|^2 E_- + B_s^{(2)}|E_2|^2 E_-
\Big].\label{eq:Pminus-direct}
\end{align}

Thus, the $\chi^{(3)}_\mathrm{eff}$ in circular basis is diagonal, and the difference between the elements is (we omit nonlinear terms for probe, $E^{(s)}$):
\begin{equation}
    \chi_{++} - \chi_{--} = |E_1|^2 (B_s^{(1)}-C_s^{(1)}) - |E_2|^2 (B_s^{(2)}-C_s^{(2)})
\end{equation}
So, we can conlude that if $|E_1| = |E_2|$ or $B_s^{(m)} = C_s^{(m)}$, there is no Faraday rotation.

\subsection{Probe sidebands at \(\Omega_\pm=\omega_s\pm\Delta\)}
From the isotropic sideband formula (with distinct coefficients for the \emph{sideband-generation} tuples),
\begin{align}
\mathbf{P}^{(3)}(\Omega_+)
&=\frac{3}{4}\eps_0\Big[
A_+^{\rm sb}\,(E^{(1)}\!\cdot\!E^{(2)*})\,\mathbf{E}^{(s)}
+ B_+^{\rm sb}\,( \mathbf{E}^{(s)}\!\cdot\!E^{(2)*})\,\mathbf{E}^{(1)}
+ C_+^{\rm sb}\,( \mathbf{E}^{(s)}\!\cdot\!E^{(1)})\,\mathbf{E}^{(2)*}
\Big],\label{eq:SBplus-start}\\
\mathbf{P}^{(3)}(\Omega_-)
&=\frac{3}{4}\eps_0\Big[
A_-^{\rm sb}\,(E^{(2)}\!\cdot\!E^{(1)*})\,\mathbf{E}^{(s)}
+ B_-^{\rm sb}\,( \mathbf{E}^{(s)}\!\cdot\!E^{(1)*})\,\mathbf{E}^{(2)}
+ C_-^{\rm sb}\,( \mathbf{E}^{(s)}\!\cdot\!E^{(2)})\,\mathbf{E}^{(1)*}
\Big],\label{eq:SBminus-start}
\end{align}
with
\[
\begin{aligned}
&A_+^{\rm sb}:=A(-\Omega_+;\omega_s,\omega_1,-\omega_2),\ 
B_+^{\rm sb}:=B(-\Omega_+;\omega_s,\omega_1,-\omega_2),\ 
C_+^{\rm sb}:=C(-\Omega_+;\omega_s,\omega_1,-\omega_2),\\
&A_-^{\rm sb}:=A(-\Omega_-;\omega_s,\omega_2,-\omega_1),\ 
B_-^{\rm sb}:=B(-\Omega_-;\omega_s,\omega_2,-\omega_1),\ 
C_-^{\rm sb}:=C(-\Omega_-;\omega_s,\omega_2,-\omega_1).
\end{aligned}
\]
Using \(E^{(1)}\!\cdot\!E^{(2)*}=0\), \( (\mathbf{E}^{(s)}\!\cdot\!E^{(2)*})=E_-E_2^* \), \( (\mathbf{E}^{(s)}\!\cdot\!E^{(1)})=E_-E_1 \), and
\(\mathbf{E}^{(1)}\parallel e^+,\ \mathbf{E}^{(2)*}\parallel e^+\),
both \eqref{eq:SBplus-start} contributions align with \(e^+\):
\[
\boxed{\
\mathbf{P}^{(3)}(\Omega_+)=\frac{3}{4}\eps_0\,(B_+^{\rm sb}+C_+^{\rm sb})\,E_1 E_2^*\,E_-\,\mathbf{e}^{+}.\
}
\]
Similarly, using \( (\mathbf{E}^{(s)}\!\cdot\!E^{(1)*})=E_+E_1^*,\ (\mathbf{E}^{(s)}\!\cdot\!E^{(2)})=E_+E_2 \) and
\(\mathbf{E}^{(2)}\parallel e^-,\ \mathbf{E}^{(1)*}\parallel e^-\), both terms in \eqref{eq:SBminus-start} align with \(e^-\):
\[
\boxed{\
\mathbf{P}^{(3)}(\Omega_-)=\frac{3}{4}\eps_0\,(B_-^{\rm sb}+C_-^{\rm sb})\,E_2 E_1^*\,E_+\,\mathbf{e}^{-}.\
}
\]

\paragraph{From sideband polarization to field.}
Let \(\mathcal{G}_{ja}(\Omega)\) be the linear Green tensor. Define its circular projections
\[
G_{++}(\Omega):=e^{(+)\!*}_j\,\mathcal{G}_{ja}(\Omega)\,e^{(+)}_a,\qquad
G_{--}(\Omega):=e^{(-)\!*}_j\,\mathcal{G}_{ja}(\Omega)\,e^{(-)}_a,
\]
(i.e., \(U^\dagger\mathcal{G}U\) and pick diagonal \(\pm\) elements). Then the generated sideband \emph{fields} are
\[
\boxed{\
\mathbf{E}^{(s+)}=G_{++}(\Omega_+)\,\mathbf{P}^{(3)}(\Omega_+),\qquad
\mathbf{E}^{(s-)}=G_{--}(\Omega_-)\,\mathbf{P}^{(3)}(\Omega_-).\
}
\]

\subsection{Cascaded back-mixing to \(\omega_s\)}
The sidebands mix with the \emph{opposite} pump pair:
\[
(\Omega_+)+\omega_2-\omega_1=\omega_s,\qquad
(\Omega_-)+\omega_1-\omega_2=\omega_s.
\]
The cascaded polarization has two arms, each with its own \emph{mixing} coefficients
\[
\begin{aligned}
&B_+^{\rm mx}:=B(-\omega_s;\Omega_+,\omega_2,-\omega_1),\quad
C_+^{\rm mx}:=C(-\omega_s;\Omega_+,\omega_2,-\omega_1),\\
&B_-^{\rm mx}:=B(-\omega_s;\Omega_-,\omega_1,-\omega_2),\quad
C_-^{\rm mx}:=C(-\omega_s;\Omega_-,\omega_1,-\omega_2).
\end{aligned}
\]
Using the isotropic contraction rules, the surviving terms are
\begin{align}
\mathbf{P}^{(3),\mathrm{casc}}(\omega_s)
&=\frac{3}{4}\eps_0\Big\{
B_+^{\rm mx}\,(\mathbf{E}^{(s+)}\!\cdot\!E^{(1)*})\,\mathbf{E}^{(2)}
+ C_+^{\rm mx}\,(\mathbf{E}^{(s+)}\!\cdot\!E^{(2)})\,\mathbf{E}^{(1)*}
\nonumber\\[-2pt]
&\hspace{6.2em}
+\,B_-^{\rm mx}\,(\mathbf{E}^{(s-)}\!\cdot\!E^{(2)*})\,\mathbf{E}^{(1)}
+ C_-^{\rm mx}\,(\mathbf{E}^{(s-)}\!\cdot\!E^{(1)})\,\mathbf{E}^{(2)*}
\Big\}.
\end{align}
Since \(\mathbf{E}^{(s+)}\parallel e^+\), \(E^{(1)*}\parallel e^-\), \(E^{(2)}\parallel e^-\) and
\(\mathbf{E}^{(s-)}\parallel e^-\), \(E^{(2)*}\parallel e^+\), \(E^{(1)}\parallel e^+\),
all four dot products are unity and the \(A\)-channel vanishes. Substituting the sideband fields gives a diagonal correction in the circular basis:
\[
\boxed{\
\begin{aligned}
\mathbf{P}^{(3),\mathrm{casc}}(\omega_s)
&=\Big(\frac{3}{4}\eps_0\Big)^{\!2}\,
|E_1|^2|E_2|^2
\Big[(B_+^{\rm mx})(B_+^{\rm sb}+C_+^{\rm sb})\,G_{++}(\Omega_+)\,E_-\,\mathbf{e}^-
\\[-2pt]
&\qquad\qquad\qquad\qquad\qquad
+(B_-^{\rm mx})(B_-^{\rm sb}+C_-^{\rm sb})\,G_{--}(\Omega_-)\,E_+\,\mathbf{e}^+ \Big]
\\[2pt]
&\quad+\Big(\frac{3}{4}\eps_0\Big)^{\!2}\,
|E_1|^2|E_2|^2
\Big[(C_+^{\rm mx})(B_+^{\rm sb}+C_+^{\rm sb})\,G_{++}(\Omega_+)\,E_-\,\mathbf{e}^-
\\[-2pt]
&\qquad\qquad\qquad\qquad\qquad
+(C_-^{\rm mx})(B_-^{\rm sb}+C_-^{\rm sb})\,G_{--}(\Omega_-)\,E_+\,\mathbf{e}^+ \Big].
\end{aligned}}
\]
Equivalently, grouping \(B\) and \(C\) on each arm:
\[
\boxed{\
\begin{aligned}
\mathbf{P}^{(3),\mathrm{casc}}(\omega_s)
&=\Big(\frac{3}{4}\eps_0\Big)^{\!2}\,|E_1|^2|E_2|^2
\Big[
\underbrace{(B_-^{\rm mx}+C_-^{\rm mx})(B_-^{\rm sb}+C_-^{\rm sb})\,G_{--}(\Omega_-)}_{\displaystyle \Pi_-(\Omega_-)}\,E_+\,\mathbf{e}^+
\\[-2pt]
&\qquad\qquad\qquad\qquad\qquad
+\underbrace{(B_+^{\rm mx}+C_+^{\rm mx})(B_+^{\rm sb}+C_+^{\rm sb})\,G_{++}(\Omega_+)}_{\displaystyle \Pi_+(\Omega_+)}\,E_-\,\mathbf{e}^-
\Big].
\end{aligned}}
\]
If we substitute the sideband fields (for TCMT), we get:
\begin{equation}
\begin{aligned}
\mathbf{P}^{(3),\mathrm{casc}}(\omega_s)
&=\frac{3}{4}\eps_0\,|E_1|^2|E_2|^2
\Big[
    (B_-^{\rm mx}+C_-^{\rm mx})\,E_{s-}^{(-)}\,\mathbf{e}^+ + (B_+^{\rm mx}+C_+^{\rm mx}) E_{s+}^{(+)}\,\mathbf{e}^-
\Big].
\end{aligned}
\end{equation}

\subsection{Total probe polarization at \(\omega_s\)}
Writing \(\mathbf{P}(\omega_s)=P^{\rm tot}_+\,\mathbf{e}^+ + P^{\rm tot}_-\,\mathbf{e}^-\), we have
\[
\boxed{\
\begin{aligned}
P^{\rm tot}_+&=P_+ \;+\;\Big(\frac{3}{4}\eps_0\Big)^{\!2}|E_1|^2|E_2|^2\,\Pi_-(\Omega_-)\,E_+,
\\[2pt]
P^{\rm tot}_-&=P_- \;+\;\Big(\frac{3}{4}\eps_0\Big)^{\!2}|E_1|^2|E_2|^2\,\Pi_+(\Omega_+)\,E_-,
\end{aligned}}
\]
with \(P_\pm\) from \eqref{eq:Pplus-direct}–\eqref{eq:Pminus-direct}. The cascaded factors
\(\Pi_\pm\) explicitly carry the \emph{distinct} coefficients from sideband generation \((B_\pm^{\rm sb},C_\pm^{\rm sb})\), the back-mixing stage \((B_\pm^{\rm mx},C_\pm^{\rm mx})\), and the circular projections of the linear Green tensor \(G_{\pm\pm}(\Omega_\pm)\).

\paragraph{Remarks.}
\begin{itemize}
\item No Kleinman symmetry was used; all \(A,B,C\) are evaluated at the specific frequency tuples indicated by the superscripts \({\rm sb}\) (sideband) and \({\rm mx}\) (mixing), and by the arm sign \(+\) or \(-\).
\item The \(A\)-channel drops out of the cascaded step because \(E^{(2)}\!\cdot E^{(1)*}=0\) for counter-rotating pumps in the transverse circular basis.
\item In gyrotropic or anisotropic backgrounds, keep the full \(2\times2\) circular block of \(\mathcal{G}\); here we used diagonal projections \(G_{\pm\pm}\).
\end{itemize}


\section{Co-rotating Pumps: Direct and Cascaded $P^{(3)}(\omega_s)$ in Isotropic Tensor Form}

\paragraph{Setup and notation}

Time convention \(e^{-\ii\omega t}\). Circular basis \(\{\mathbf e^{(+)},\mathbf e^{(-)},\mathbf e^{(z)}\}\) with
\(\mathbf e^{(+)}\!\cdot\!\mathbf e^{(-)}=1\), \(\mathbf e^{(\pm)}\!\cdot\!\mathbf e^{(\pm)}=0\), and \((\mathbf e^{(+)})^*=\mathbf e^{(-)}\).
Fields:
\[
E^{(1)}_i=E_1\,e^{(+)}_i,\qquad
E^{(2)}_i=E_2\,e^{(+)}_i,\qquad
E^{(s)}_i=E_+\,e^{(+)}_i+E_-\,e^{(-)}_i,
\]
and \(\Delta=\omega_1-\omega_2\), \(\Omega_\pm=\omega_s\pm\Delta\).

\paragraph{Isotropic third-order tensor (frequency-tuple dependent).}
For any frequency tuple \(\Xi\),
\[
\chi^{(3)}_{ijkl}(\Xi)
=A(\Xi)\,\delta_{ij}\delta_{kl}+B(\Xi)\,\delta_{ik}\delta_{jl}+C(\Xi)\,\delta_{il}\delta_{jk}.
\]
We keep \(A,B,C\) distinct for each \(\Xi\).

\subsection{Direct third-order polarization at $\omega_s$}

Start from
\begin{align}
P^{(3)}_i(\omega_s)
=\frac{3}{4}\eps_0\!\!\sum_{m\in\{s,1,2\}}\!\!
\Big[A_s^{(m)}\delta_{ij}\delta_{kl}+B_s^{(m)}\delta_{ik}\delta_{jl}+C_s^{(m)}\delta_{il}\delta_{jk}\Big]\,
E^{(s)}_j E^{(m)}_k E^{(m)*}_l,
\label{eq:P3direct-start}
\end{align}
with \(A_s^{(m)}:=A(-\omega_s;\omega_s,\omega_m,-\omega_m)\) and similarly \(B_s^{(m)},C_s^{(m)}\).

Write \(P^{(3)}_i(\omega_s)=P_+\,e^{(+)}_i+P_-\,e^{(-)}_i\). A straightforward contraction in the circular basis gives a diagonal result:
\begin{align}
\boxed{
\begin{aligned}
P_+&=\frac{3}{4}\eps_0\Big[
(A_s^{(s)}+C_s^{(s)})(|E_+|^2+|E_-|^2)\,E_+ + 2B_s^{(s)}|E_-|^2\,E_+\\
&\hspace{3.4em}
+(A_s^{(1)}+B_s^{(1)})|E_1|^2\,E_+ + (A_s^{(2)}+B_s^{(2)})|E_2|^2\,E_+\Big],\\[4pt]
P_-&=\frac{3}{4}\eps_0\Big[
(A_s^{(s)}+C_s^{(s)})(|E_+|^2+|E_-|^2)\,E_- + 2B_s^{(s)}|E_+|^2\,E_-\\
&\hspace{3.4em}
+(A_s^{(1)}+C_s^{(1)})|E_1|^2\,E_- + (A_s^{(2)}+C_s^{(2)})|E_2|^2\,E_-\Big].
\end{aligned}}
\label{eq:P3direct-final}
\end{align}

\subsection{Probe sidebands at $\Omega_\pm=\omega_s\pm\Delta$}

Using the isotropic contraction for triplets \((\omega_s,\omega_1,-\omega_2)\) and \((\omega_s,\omega_2,-\omega_1)\),
\begin{align}
\mathbf P^{(3)}(\Omega_+)
&=\tfrac{3}{4}\eps_0\Big[
A_+^{\rm sb}\,( \mathbf E^{(1)}\!\cdot\!\mathbf E^{(2)*})\,\mathbf E^{(s)}
+B_+^{\rm sb}\,( \mathbf E^{(s)}\!\cdot\!\mathbf E^{(2)*})\,\mathbf E^{(1)}
+C_+^{\rm sb}\,( \mathbf E^{(s)}\!\cdot\!\mathbf E^{(1)})\,\mathbf E^{(2)*}\Big],
\\
\mathbf P^{(3)}(\Omega_-)
&=\tfrac{3}{4}\eps_0\Big[
A_-^{\rm sb}\,( \mathbf E^{(2)}\!\cdot\!\mathbf E^{(1)*})\,\mathbf E^{(s)}
+B_-^{\rm sb}\,( \mathbf E^{(s)}\!\cdot\!\mathbf E^{(1)*})\,\mathbf E^{(2)}
+C_-^{\rm sb}\,( \mathbf E^{(s)}\!\cdot\!\mathbf E^{(2)})\,\mathbf E^{(1)*}\Big],
\end{align}
where, e.g., \(A_+^{\rm sb}:=A(-\Omega_+;\omega_s,\omega_1,-\omega_2)\), etc.

For co-rotating pumps \(\mathbf E^{(1)}\parallel\mathbf e^{(+)}\), \(\mathbf E^{(2)}\parallel\mathbf e^{(+)}\):
\[
\mathbf E^{(1)}\!\cdot\!\mathbf E^{(2)*}=E_1E_2^*,\quad
\mathbf E^{(s)}\!\cdot\!\mathbf E^{(2)*}=E_+E_2^*,\quad
\mathbf E^{(s)}\!\cdot\!\mathbf E^{(1)}=E_-E_1.
\]
Decomposing along \(\mathbf e^{(\pm)}\) gives
\begin{align}
\boxed{
\mathbf P^{(3)}(\Omega_+)
=\tfrac{3}{4}\eps_0\,E_1E_2^*
\Big[(A_+^{\rm sb}+B_+^{\rm sb})\,E_+\,\mathbf e^{(+)}
+(A_+^{\rm sb}+C_+^{\rm sb})\,E_-\,\mathbf e^{(-)}\Big],}
\label{eq:SBplus}\\
\boxed{
\mathbf P^{(3)}(\Omega_-)
=\tfrac{3}{4}\eps_0\,E_2E_1^*
\Big[(A_-^{\rm sb}+B_-^{\rm sb})\,E_+\,\mathbf e^{(+)}
+(A_-^{\rm sb}+C_-^{\rm sb})\,E_-\,\mathbf e^{(-)}\Big].}
\label{eq:SBminus}
\end{align}

\paragraph{From sideband polarization to field.}
Let the circular-basis Green block be
\[
\mathbf G(\Omega)=
\begin{pmatrix}
G_{++}(\Omega) & G_{+-}(\Omega)\\
G_{-+}(\Omega) & G_{--}(\Omega)
\end{pmatrix},
\quad
G_{\alpha\beta}(\Omega):=e^{(\alpha)\!*}_j\,\mathcal G_{ja}(\Omega)\,e^{(\beta)}_a .
\]
Define column vectors
\[
\mathbf S^{(+)}=
\begin{bmatrix}
(A_+^{\rm sb}+B_+^{\rm sb})\,E_+\\[2pt]
(A_+^{\rm sb}+C_+^{\rm sb})\,E_-
\end{bmatrix},\qquad
\mathbf S^{(-)}=
\begin{bmatrix}
(A_-^{\rm sb}+B_-^{\rm sb})\,E_+\\[2pt]
(A_-^{\rm sb}+C_-^{\rm sb})\,E_-
\end{bmatrix}.
\]
Then
\begin{align}
\begin{bmatrix}E^{(s+)}_+\\ E^{(s+)}_-\end{bmatrix}
&=\tfrac{3}{4}\eps_0\,(E_1E_2^*)\,\mathbf G(\Omega_+)\,\mathbf S^{(+)},\qquad
\begin{bmatrix}E^{(s-)}_+\\ E^{(s-)}_-\end{bmatrix}
=\tfrac{3}{4}\eps_0\,(E_2E_1^*)\,\mathbf G(\Omega_-)\,\mathbf S^{(-)}.
\label{eq:SBfields}
\end{align}


\subsection{Cascaded back-mixing to $\omega_s$}

\paragraph{Frequencies and coefficients.}
The cascaded back-mixing paths satisfy
\((\Omega_+)+\omega_2-\omega_1=\omega_s\) and \((\Omega_-)+\omega_1-\omega_2=\omega_s\).
For the \emph{mixing} stage define
\[
\begin{aligned}
&A_+^{\rm mx}:=A(-\omega_s;\Omega_+,\omega_2,-\omega_1),\quad
B_+^{\rm mx}:=B(-\omega_s;\Omega_+,\omega_2,-\omega_1),\quad
C_+^{\rm mx}:=C(-\omega_s;\Omega_+,\omega_2,-\omega_1),\\
&A_-^{\rm mx}:=A(-\omega_s;\Omega_-,\omega_1,-\omega_2),\quad
B_-^{\rm mx}:=B(-\omega_s;\Omega_-,\omega_1,-\omega_2),\quad
C_-^{\rm mx}:=C(-\omega_s;\Omega_-,\omega_1,-\omega_2).
\end{aligned}
\]

\paragraph{Upper arm (\(+\)).}
Using $E^{(2)}\!\parallel\!\mathbf e^{(+)}$, $E^{(1)*}\!\parallel\!\mathbf e^{(-)}$, the mixing at $\Omega_+$ preserves the sideband’s circular index within each channel, so the two channels \((A,B)\) add on the \(+\) component and \((A,C)\) add on the \(-\) component:
\[
\begin{aligned}
P^{(3),\mathrm{casc}}_{(+),+}
&=\tfrac{3}{4}\varepsilon_0\,\big(A_+^{\rm mx}\!+B_+^{\rm mx}\big)\,E_2E_1^*\,E^{(s+)}_{+},\\
P^{(3),\mathrm{casc}}_{(+),-}
&=\tfrac{3}{4}\varepsilon_0\,\big(A_+^{\rm mx}\!+C_+^{\rm mx}\big)\,E_2E_1^*\,E^{(s+)}_{-}.
\end{aligned}
\]

\paragraph{Lower arm (\(-\)).}
Using $E^{(1)}\!\parallel\!\mathbf e^{(+)}$, $E^{(2)*}\!\parallel\!\mathbf e^{(-)}$, the mixing at $\Omega_-$ gives
\[
\begin{aligned}
P^{(3),\mathrm{casc}}_{(-),+}
&=\tfrac{3}{4}\varepsilon_0\,\big(A_-^{\rm mx}\!+B_-^{\rm mx}\big)\,E_1E_2^*\,E^{(s-)}_{+},\\
P^{(3),\mathrm{casc}}_{(-),-}
&=\tfrac{3}{4}\varepsilon_0\,\big(A_-^{\rm mx}\!+C_-^{\rm mx}\big)\,E_1E_2^*\,E^{(s-)}_{-}.
\end{aligned}
\]

\paragraph{Collected result.}
Summing the two arms and projecting on $\{\mathbf e^{(+)},\mathbf e^{(-)}\}$ gives
\[
\boxed{
\begin{aligned}
P^{(3),\mathrm{casc}}_{+}
&=(A_+^{\rm mx}\!+B_+^{\rm mx})\,E_2E_1^{*}\,E^{(s+)}_{+}
+(A_-^{\rm mx}\!+B_-^{\rm mx})\,E_1E_2^{*}\,E^{(s-)}_{+},\\[2pt]
P^{(3),\mathrm{casc}}_{-}
&=(A_+^{\rm mx}\!+C_+^{\rm mx})\,E_2E_1^{*}\,E^{(s+)}_{-}
+(A_-^{\rm mx}\!+C_-^{\rm mx})\,E_1E_2^{*}\,E^{(s-)}_{-}.
\end{aligned}}
\]

\paragraph{Cascaded back-mixing with sideband fields included.}
From \eqref{eq:SBfields},
\[
\begin{bmatrix}E^{(s+)}_+\\ E^{(s+)}_-\end{bmatrix}
=\tfrac{3}{4}\varepsilon_0\,(E_1E_2^*)\,\mathbf G(\Omega_+)\,\mathbf S^{(+)},\qquad
\begin{bmatrix}E^{(s-)}_+\\ E^{(s-)}_-\end{bmatrix}
=\tfrac{3}{4}\varepsilon_0\,(E_2E_1^*)\,\mathbf G(\Omega_-)\,\mathbf S^{(-)}.
\]
Using the collected mixing relations
\[
\begin{aligned}
P^{(3),\mathrm{casc}}_{+}
&=(A_+^{\rm mx}\!+B_+^{\rm mx})\,E_2E_1^{*}\,E^{(s+)}_{+}
+(A_-^{\rm mx}\!+B_-^{\rm mx})\,E_1E_2^{*}\,E^{(s-)}_{+},\\
P^{(3),\mathrm{casc}}_{-}
&=(A_+^{\rm mx}\!+C_+^{\rm mx})\,E_2E_1^{*}\,E^{(s+)}_{-}
+(A_-^{\rm mx}\!+C_-^{\rm mx})\,E_1E_2^{*}\,E^{(s-)}_{-},
\end{aligned}
\]
we obtain the compact block form
\[
\boxed{
\begin{bmatrix}
P^{\mathrm{casc}}_{+}\\[2pt]
P^{\mathrm{casc}}_{-}
\end{bmatrix}
=\Big(\tfrac{3}{4}\varepsilon_0\Big)^{\!2}
|E_1|^2|E_2|^2\!
\left[
\mathbf D^{(+)} \,\mathbf G(\Omega_+)\,\mathbf S^{(+)}
+
\mathbf D^{(-)} \,\mathbf G(\Omega_-)\,\mathbf S^{(-)}
\right]\!,
}
\]
where
\[
\mathbf S^{(+)}=
\begin{bmatrix}
(A_+^{\rm sb}\!+B_+^{\rm sb})\,E_+\\[2pt]
(A_+^{\rm sb}\!+C_+^{\rm sb})\,E_-
\end{bmatrix},\qquad
\mathbf S^{(-)}=
\begin{bmatrix}
(A_-^{\rm sb}\!+B_-^{\rm sb})\,E_+\\[2pt]
(A_-^{\rm sb}\!+C_-^{\rm sb})\,E_-
\end{bmatrix}
\]
and
\begin{equation}
  \mathbf D^{(+)} =  \begin{pmatrix}
A_+^{\rm mx}\!+B_+^{\rm mx} & 0\\[2pt]
0 & A_+^{\rm mx}\!+C_+^{\rm mx}
\end{pmatrix},\qquad
  \mathbf D^{(-)} =  \begin{pmatrix}
A_-^{\rm mx}\!+B_-^{\rm mx} & 0\\[2pt]
0 & A_-^{\rm mx}\!+C_-^{\rm mx}
\end{pmatrix},\qquad
\end{equation}

\paragraph{Scalar expansions.}
Writing $\mathbf G(\Omega)=\begin{psmallmatrix}G_{++}&G_{+-}\\G_{-+}&G_{--}\end{psmallmatrix}$,
\[
\boxed{
\begin{aligned}
P^{(3),\mathrm{casc}}_{+}
&=\Big(\tfrac{3}{4}\varepsilon_0\Big)^{\!2}|E_1|^2|E_2|^2\Big[
(A_+^{\rm mx}\!+B_+^{\rm mx})\big(G_{++}(\Omega_+)\,S^{(+)}_1+G_{+-}(\Omega_+)\,S^{(+)}_2\big)\\[-2pt]
&\hspace{6.2cm}+(A_-^{\rm mx}\!+B_-^{\rm mx})\big(G_{++}(\Omega_-)\,S^{(-)}_1+G_{+-}(\Omega_-)\,S^{(-)}_2\big)\Big],\\[4pt]
P^{(3),\mathrm{casc}}_{-}
&=\Big(\tfrac{3}{4}\varepsilon_0\Big)^{\!2}|E_1|^2|E_2|^2\Big[
(A_+^{\rm mx}\!+C_+^{\rm mx})\big(G_{-+}(\Omega_+)\,S^{(+)}_1+G_{--}(\Omega_+)\,S^{(+)}_2\big)\\[-2pt]
&\hspace{6.2cm}+(A_-^{\rm mx}\!+C_-^{\rm mx})\big(G_{-+}(\Omega_-)\,S^{(-)}_1+G_{--}(\Omega_-)\,S^{(-)}_2\big)\Big],
\end{aligned}}
\]
with
\[
S^{(+)}_1=(A_+^{\rm sb}\!+B_+^{\rm sb})\,E_+,\quad
S^{(+)}_2=(A_+^{\rm sb}\!+C_+^{\rm sb})\,E_-,\quad
S^{(-)}_1=(A_-^{\rm sb}\!+B_-^{\rm sb})\,E_+,\quad
S^{(-)}_2=(A_-^{\rm sb}\!+C_-^{\rm sb})\,E_-.
\]

\paragraph{Notes.}
(i) The pump products combine as $(E_2E_1^*)(E_1E_2^*)=|E_1|^2|E_2|^2$, so all residual phase sensitivity resides in $\mathbf G(\Omega_\pm)$ and the probe $E_\pm$. 
(ii) Off-diagonal $G_{\pm\mp}$ terms are the channels that convert circular components and are the ones that ultimately drive polarization rotation in the probe.

\subsection{Total probe polarization at $\omega_s$}

Let \(P^{\rm dir}_\pm\) denote the direct terms from \eqref{eq:P3direct-final}. The full result is
\[
\boxed{
\begin{bmatrix}
P^{\rm tot}_+\\[2pt]
P^{\rm tot}_-
\end{bmatrix}
=
\begin{bmatrix}
P^{\rm dir}_+\\[2pt]
P^{\rm dir}_-
\end{bmatrix}
+\Big(\tfrac{3}{4}\eps_0\Big)^{\!2}
\left[
\mathbf D^{(+)}\,\mathbf G(\Omega_+)\,\mathbf S^{(+)}
+\mathbf D^{(-)}\,\mathbf G(\Omega_-)\,\mathbf S^{(-)}
\right].
}
\]

\paragraph{Remarks.}
\begin{itemize}
\item No Kleinman symmetry was used; every \(A,B,C\) carries its own frequency tuple:
\(\{ -\omega_s;\omega_s,\omega_m,-\omega_m\}\) for the direct term,
\(\{-\Omega_\pm;\omega_s,\omega_{1/2},-\omega_{2/1}\}\) for sideband generation,
and \(\{-\omega_s;\Omega_\pm,\omega_{2/1},-\omega_{1/2}\}\) for back-mixing.
\item In homogeneous isotropic propagation, \(\mathbf G(\Omega_\pm)\) reduces to a diagonal scalar times \(\mathbf I_{2}\), simplifying the cascaded matrices.
\item The direct term is strictly diagonal in \(\{+,-\}\) (no mixing); any \(+\leftrightarrow-\) coupling at \(\omega_s\) here originates from the cascaded pathway and/or off-diagonal \(G_{\alpha\beta}\).
\end{itemize}

\end{document}
