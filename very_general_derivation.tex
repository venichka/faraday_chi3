\documentclass[11pt]{article}
\usepackage[margin=1in]{geometry}
\usepackage{amsmath,amssymb,bm}
\usepackage{physics}
\usepackage{mathtools}

\newcommand{\e}{\mathrm{e}}
\newcommand{\ii}{\mathrm{i}}
\newcommand{\eps}{\varepsilon}

\begin{document}

\begin{center}
  {\LARGE \textbf{Probe, Sidebands, and Cascaded Back-Action at \(\omega_s\)\\[2pt]
  in the General \(\chi^{(3)}_{ijkl}\) Tensor Formalism}}
\end{center}

\section*{Conventions and fields}
Time convention \( \e^{-\ii\omega t} \). The (analytic) electric field is written as
\begin{equation}
  E_i(t)=\frac12\sum_{q}\!\Big(E^{(q)}_i \e^{-\ii\omega_q t}+E^{(q)*}_i \e^{+\ii\omega_q t}\Big),
\end{equation}
with a weak probe at \(\omega_s\) and two pumps at \(\omega_1,\omega_2\) (the latter will later be specialized to \(\omega_{1,2}=\omega_p\pm \Delta/2\) only to name their beat \(\Delta\); none of the derivations below assumes isotropy or a polarization basis).
We use Einstein summation over repeated Cartesian indices \(i,j,k,l,\dots\).

\paragraph{Third-order polarization (frequency domain).}
For any target frequency \(\Omega\),
\begin{equation}
  P^{(3)}_i(\Omega)
  = \eps_0\,\chi^{(3)}_{ijkl}\!\big(-\Omega;\omega_\alpha,\omega_\beta,\omega_\gamma\big)\,
    E^{(\alpha)}_j E^{(\beta)}_k E^{(\gamma)}_l\;
    \delta(\omega_\alpha+\omega_\beta+\omega_\gamma-\Omega).
  \label{eq:P3-general}
\end{equation}
It is convenient to use the \emph{intrinsic-permutation symmetrization} over the last three slots (indices/frequencies):
\begin{equation}
  \bar\chi^{(3)}_{ijkl}\!\big(-\Omega;\omega_a,\omega_b,\omega_c\big)
  :=\frac{1}{3!}\!\!\sum_{\pi\in S_3}
  \chi^{(3)}_{i\,\pi(jkl)}\!\big(-\Omega;\pi[\omega_a,\omega_b,\omega_c]\big).
  \label{eq:chi-bar}
\end{equation}
When extracting \(\Omega\) components from an analytic field (half-field decomposition), the usual degeneracy factor \(3/4\) appears for the fundamental-like channels considered below.

\section*{A. Direct third-order polarization at \(\bm{\omega_s}\)}
Retain only terms \emph{linear in the probe} \(E^{(s)}\). The contributing triplets are
\((\omega_s,\omega_m,-\omega_m)\) with \(m\in\{s,1,2\}\). Hence,
\begin{align}
  P^{(3)}_i(\omega_s)
  &= \frac{3}{4}\,\eps_0
     \sum_{m\in\{s,1,2\}}
     \bar\chi^{(3)}_{ijkl}\!\big(-\omega_s;\omega_s,\omega_m,-\omega_m\big)\,
     E^{(s)}_j\,E^{(m)}_k E^{(m)*}_l.
  \label{eq:P3-direct}
\end{align}
Equation \eqref{eq:P3-direct} contains probe SPM (\(m=s\)) and XPM from each pump (\(m=1,2\)). No basis or material symmetry has been assumed.

\section*{B. Probe sidebands at \(\bm{\omega_s\pm\Delta}\)}
Let \(\Delta_{mn}:=\omega_m-\omega_n\). The sideband frequencies that are \emph{linear in the probe} are
\begin{equation}
  \Omega_{mn}^{(+)} = \omega_s+\Delta_{mn}=\omega_s+\omega_m-\omega_n, \qquad
  \Omega_{mn}^{(-)} = \omega_s-\Delta_{mn}=\omega_s-\omega_m+\omega_n .
\end{equation}
From \eqref{eq:P3-general} with the contributing triplets \((\omega_s,\omega_m,-\omega_n)\) and permutations, we obtain
\begin{align}
  P^{(3)}_i\!\big(\Omega_{mn}^{(+)}\big)
  &= \frac{3}{4}\,\eps_0\,
     \bar\chi^{(3)}_{ijkl}\!\big(-\Omega_{mn}^{(+)};\omega_s,\omega_m,-\omega_n\big)\,
     E^{(s)}_j\,E^{(m)}_k E^{(n)*}_l,
  \label{eq:P3-sideband-plus}
  \\
  P^{(3)}_i\!\big(\Omega_{mn}^{(-)}\big)
  &= \frac{3}{4}\,\eps_0\,
     \bar\chi^{(3)}_{ijkl}\!\big(-\Omega_{mn}^{(-)};\omega_s,-\omega_m,\omega_n\big)\,
     E^{(s)}_j\,E^{(m)*}_k E^{(n)}_l.
  \label{eq:P3-sideband-minus}
\end{align}
Equations \eqref{eq:P3-sideband-plus}–\eqref{eq:P3-sideband-minus} are fully tensorial and basis free.

\section*{C. Linear propagation from polarization to sideband fields}
Let \(\mathcal{G}_{ja}(\Omega)\) denote the \emph{linear} Green tensor at frequency \(\Omega\) that maps polarization to field (this includes local material response and propagation/phase matching). Then the sideband fields generated by \eqref{eq:P3-sideband-plus}–\eqref{eq:P3-sideband-minus} are
\begin{align}
  E^{(s+)}_j\!\big(\Omega_{mn}^{(+)}\big)
  &= \mathcal{G}_{ja}\!\big(\Omega_{mn}^{(+)}\big)\;
     P^{(3)}_a\!\big(\Omega_{mn}^{(+)}\big),
  \label{eq:Es-plus}
  \\
  E^{(s-)}_j\!\big(\Omega_{mn}^{(-)}\big)
  &= \mathcal{G}_{ja}\!\big(\Omega_{mn}^{(-)}\big)\;
     P^{(3)}_a\!\big(\Omega_{mn}^{(-)}\big).
  \label{eq:Es-minus}
\end{align}

\section*{D. Cascaded back-action to \(\bm{\omega_s}\)}
The generated sidebands can mix with the \emph{opposite} pump pair to return energy to \(\omega_s\) because
\[
  \big(\omega_s+\omega_m-\omega_n\big)+\omega_n-\omega_m=\omega_s, \qquad
  \big(\omega_s-\omega_m+\omega_n\big)+\omega_m-\omega_n=\omega_s.
\]
Therefore, the cascaded third-order polarization (again linear in \(E^{(s)}\)) is
\begin{align}
  P^{(3),\mathrm{casc}}_i(\omega_s)
  &= \frac{3}{4}\,\eps_0
     \sum_{m,n}\Bigg[
       \bar\chi^{(3)}_{ijkl}\!\big(-\omega_s;\Omega_{mn}^{(+)},\omega_n,-\omega_m\big)\;
       E^{(s+)}_j\!\big(\Omega_{mn}^{(+)}\big)\; E^{(n)}_k E^{(m)*}_l
       \notag\\[-2pt]
       &\hspace{6.4em}
       +\;
       \bar\chi^{(3)}_{ijkl}\!\big(-\omega_s;\Omega_{mn}^{(-)},\omega_m,-\omega_n\big)\;
       E^{(s-)}_j\!\big(\Omega_{mn}^{(-)}\big)\; E^{(m)}_k E^{(n)*}_l
     \Bigg].
  \label{eq:P3-casc-raw}
\end{align}
Substituting \eqref{eq:Es-plus}–\eqref{eq:Es-minus} and then
\eqref{eq:P3-sideband-plus}–\eqref{eq:P3-sideband-minus} yields a compact \emph{fifth-order} effective response acting on the probe:
\begin{align}
  P^{(3),\mathrm{casc}}_i(\omega_s)
  &= \Big(\frac{3}{4}\,\eps_0\Big)^2
     \sum_{m,n}\Bigg\{
       \underbrace{\bar\chi^{(3)}_{ijkl}\!\big(-\omega_s;\Omega_{mn}^{(+)},\omega_n,-\omega_m\big)\,
       \mathcal{G}_{ja}\!\big(\Omega_{mn}^{(+)}\big)\,
       \bar\chi^{(3)}_{abcd}\!\big(-\Omega_{mn}^{(+)};\omega_s,\omega_m,-\omega_n\big)}_{\displaystyle
       \mathcal{K}^{(+)}_{i b k l c d}(\omega_s;\Omega_{mn}^{(+)})} \\
       &\hspace{8.2em} E^{(s)}_b\, E^{(m)}_c E^{(n)*}_d\, E^{(n)}_k E^{(m)*}_l
       \notag\\[-2pt]
       &\hspace{8.2em}
       +\;
       \underbrace{\bar\chi^{(3)}_{ijkl}\!\big(-\omega_s;\Omega_{mn}^{(-)},\omega_m,-\omega_n\big)\,
       \mathcal{G}_{ja}\!\big(\Omega_{mn}^{(-)}\big)\,
       \bar\chi^{(3)}_{abcd}\!\big(-\Omega_{mn}^{(-)};\omega_s,-\omega_m,\omega_n\big)}_{\displaystyle
       \mathcal{K}^{(-)}_{i b k l c d}(\omega_s;\Omega_{mn}^{(-)})}\\
       &\hspace{8.2em} E^{(s)}_b\, E^{(m)*}_c E^{(n)}_d\, E^{(m)}_k E^{(n)*}_l
     \Bigg\}.
  \label{eq:P3-casc-compact}
\end{align}
It is natural to introduce pump dyadics
\(
  M^{(mn)}_{kl} := E^{(n)}_k E^{(m)*}_l
\)
and probe vectors \(E^{(s)}_b\) to exhibit the effective rank-2 response felt by the probe:
\begin{align}
  P^{(3),\mathrm{casc}}_i(\omega_s)
  &= \Theta^{(5)}_{i b}(\omega_s)\; E^{(s)}_b, \\
  \Theta^{(5)}_{i b}(\omega_s)
  &= \Big(\frac{3}{4}\,\eps_0\Big)^2
    \sum_{m,n}\!\Big[
      \mathcal{K}^{(+)}_{i b k l c d}(\omega_s;\Omega_{mn}^{(+)})\, M^{(mn)}_{kl}\, M^{(mn)}_{cd}
      + \mathcal{K}^{(-)}_{i b k l c d}(\omega_s;\Omega_{mn}^{(-)})\, M^{(nm)}_{kl}\, M^{(nm)}_{cd}
    \Big].
  \label{eq:Theta5}
\end{align}
Equation \eqref{eq:Theta5} is a \emph{general} (basis-free) expression for the cascaded, pump-intensity-quadratic correction at \(\omega_s\) produced by \(\chi^{(3)}\) sideband generation plus linear propagation plus \(\chi^{(3)}\) back-mixing.

\section*{E. Total probe polarization at \(\bm{\omega_s}\)}
Combining the direct third-order response \eqref{eq:P3-direct} and the cascaded contribution \eqref{eq:Theta5} gives
\begin{align}
  P_i(\omega_s)
  &= P^{(3)}_i(\omega_s)\;+\;P^{(3),\mathrm{casc}}_i(\omega_s)
   \notag\\[2pt]
  &= \frac{3}{4}\,\eps_0
     \sum_{m\in\{s,1,2\}}
     \bar\chi^{(3)}_{ijkl}\!\big(-\omega_s;\omega_s,\omega_m,-\omega_m\big)\,
     E^{(s)}_j\,E^{(m)}_k E^{(m)*}_l
     \;+\; \Theta^{(5)}_{i b}(\omega_s)\,E^{(s)}_b .
  \label{eq:P-total}
\end{align}
The first term is the standard SPM/XPM tensor response (order \(\chi^{(3)}E_s|E|^2\)); the second is an \emph{effective fifth-order} correction (order \(\chi^{(3)}\mathcal{G}\chi^{(3)}\,E_s|E_1|^2|E_2|^2\)) whose size and sign are governed by the full tensor contractions and the linear Green tensor at the sideband frequencies.

\paragraph{Remarks.}
\begin{itemize}
  \item No assumptions about isotropy, Kleinman symmetry, or polarization basis were made. All frequency arguments are explicit.
  \item The sideband routes \(\Omega_{mn}^{(\pm)}\) may include \(m=n\) (collapsing to the fundamental and reproducing the direct term) or distinct pumps \(m\neq n\) (true probe satellites at \(\omega_s\pm\Delta_{mn}\)).
  \item A concrete propagation model (bulk plane waves, guided modes, etc.) specifies \(\mathcal{G}_{ja}(\Omega)\) and thus exposes explicit phase-mismatch denominators inside \(\Theta^{(5)}_{i b}(\omega_s)\).
\end{itemize}

\end{document}
