\documentclass[11pt]{article}
\usepackage[a4paper,margin=1in]{geometry}
\usepackage{amsmath,amssymb,amsfonts,bm}
\usepackage{physics}
\usepackage{hyperref}
\usepackage{titlesec}
\usepackage{graphicx}
\usepackage{mathtools}

\titleformat{\section}{\large\bfseries}{\thesection}{0.6em}{}
\titleformat{\subsection}{\normalsize\bfseries}{\thesubsection}{0.6em}{}
\title{Faraday Rotation from $\chi^{(3)}$ in an Isotropic Film}

\author{}
\date{\today}

\begin{document}
\maketitle

\begin{abstract}
We derive the effective third-order susceptibility $\underline{\underline{\chi}}^{\,\mathrm{eff}}(\omega_s)$ experienced by a weak, linearly polarized signal at frequency $\omega_s$ propagating through a homogeneous, centrosymmetric, isotropic film in the $xy$ plane. The medium is driven by two
strong circularly polarized pumps: pump~1 (helicity $\sigma^+$) at $\omega_{p+}=\omega_p+\Delta/2$, and pump~2 (helicity $\sigma^-$) at $\omega_{p-}=\omega_p-\Delta/2$. Keeping all phase factors and frequency selection rules explicit, we reduce the isotropic rank-4 tensor $\chi^{(3)}_{ijkl}$ to
scalar and axial invariants, construct $\underline{\underline{\chi}}^{\,\mathrm{eff}}$, and extract the circular eigenindices $n_\pm$ and the Faraday rotation angle $\theta_F$ at the carrier. Cross-tone (beating) terms are shown to generate sidebands at $\omega_s\pm\Delta$ and are not part of $\underline{\underline{\chi}}^{\,\mathrm{eff}}(\omega_s)$.
\end{abstract}

\section{Fields, Frequencies, and Conventions}
We work with analytic electric fields; physical fields are the real parts (let's not forget:). Cartesian indices $i,j,k,l\in\{x,y,z\}$ are summed when repeated (again, good to remember:).

\paragraph{Signal (probe).} A weak probe (or signal) at $\omega_s$:
\begin{equation}
E_{s,i}(t)=E_{s,i}(\omega_s)\,e^{-i\omega_s t}.
\end{equation}

\paragraph{Pumps.} Two strong tones
\begin{align}
E^{(1)}_{k}(t)&=E^{(1)}_{k}(\omega_{p+})\,e^{-i\omega_{p+} t},\qquad \omega_{p+}=\omega_p+\frac{\Delta}{2},\\
E^{(2)}_{k}(t)&=E^{(2)}_{k}(\omega_{p-})\,e^{-i\omega_{p-} t},\qquad \omega_{p-}=\omega_p-\frac{\Delta}{2}.
\end{align}
In the frequency domain,
\begin{equation}
E_{p,k}(\omega)=E^{(1)}_k(\omega)\,+\,E^{(2)}_k(\omega),\qquad
E^{(j)}_k(\omega)=E^{(j)}_k(\omega_{p j})\,\delta(\omega-\omega_{p j}).
\end{equation}

\paragraph{Circular basis.} Define $\hat{\mathbf e}_\pm=(\hat{\mathbf x}\mp i\hat{\mathbf y})/\sqrt{2}$, with $\hat{\mathbf e}_\pm\!\cdot\!\hat{\mathbf e}_\pm=1$ and $\hat{\mathbf e}_+\!\cdot\!\hat{\mathbf e}_-=0$. For the specific case considered later we set
\begin{equation}
\mathbf E^{(1)}=E^{(1)}_{+}\,\hat{\mathbf e}_+,\qquad
\mathbf E^{(2)}=E^{(2)}_{-}\,\hat{\mathbf e}_-.
\end{equation}

\section{Third-Order Polarization at a Specified Output Frequency}
The frequency-resolved third-order polarization is
\begin{equation}\label{eq:P3_general}
P^{(3)}_i(\omega)=\varepsilon_0\!\!\sum_{\omega_1,\omega_2,\omega_3}\!
\chi^{(3)}_{ijkl}\!\left(-\omega;\omega_1,\omega_2,\omega_3\right)
E_j(\omega_1)E_k(\omega_2)E_l(\omega_3)\,
\delta\!\big(\omega-\omega_1-\omega_2-\omega_3\big).
\end{equation}
For cross-phase modulation (XPM) of the signal at $\omega=\omega_s$, we keep the pathway $(-\omega_s;\,\omega_s,\omega_a,-\omega_b)$:
\begin{equation}\label{eq:P3_signal}
P^{(3)}_i(\omega_s)=\varepsilon_0\sum_{\omega_a,\omega_b}
\chi^{(3)}_{ijkl}\!\left(-\omega_s;\omega_s,\omega_a,-\omega_b\right)
E_{s,j}(\omega_s)\,E_{p,k}(\omega_a)\,E^\ast_{p,l}(\omega_b)\,
\delta(\omega_a-\omega_b).
\end{equation}
The delta function enforces $\omega_a=\omega_b$, so \emph{only same-tone pump pairs contribute at the carrier $\omega_s$}. Cross-tone products with $\omega_a\neq\omega_b$ produce time factors $e^{\mp i\Delta t}$ and appear at $\omega_s\pm\Delta$ (sidebands), not at $\omega_s$. With two tones $\omega_{p+}$ and $\omega_{p-}$, Eq.~\eqref{eq:P3_signal} reduces to
\begin{equation}\label{eq:P3_two_tones}
P^{(3)}_i(\omega_s)=\varepsilon_0\sum_{m=1}^2
\chi^{(3,j)}_{ijkl}\,
E_{s,j}(\omega_s)\,E^{(j)}_{k}(\omega_{p j})\,E^{(j)\,\ast}_{l}(\omega_{p j}),
\quad
\chi^{(3,j)}_{ijkl}\equiv\chi^{(3)}_{ijkl}\!\left(-\omega_s;\omega_s,\omega_{p j},-\omega_{p j}\right).
\end{equation}

\section{Isotropic $\chi^{(3)}$ and Cartesian Contractions}
For a homogeneous isotropic centrosymmetric medium (off resonance), the rank-4 tensor decomposes into three scalar coefficients (often denoted $\chi_A,\chi_B,\chi_C$) built from Kronecker deltas:
\begin{equation}\label{eq:isotropic_decomp}
\chi^{(3,j)}_{ijkl}
=\chi_A^{(j)}\,\delta_{ij}\delta_{kl}
+\chi_B^{(j)}\,\delta_{ik}\delta_{jl}
+\chi_C^{(j)}\,\delta_{il}\delta_{jk}.
\end{equation}
Insert \eqref{eq:isotropic_decomp} into \eqref{eq:P3_two_tones} and contract indices (Einstein summation understood). For each tone $j$:
\begin{align}
\text{(A)}:\quad & \chi_A^{(j)}\,\delta_{ij}\delta_{kl}\,E_{s,j}\,E^{(j)}_{k}E^{(j)\,\ast}_{l}
=\chi_A^{(j)}\,(\mathbf E^{(j)}\!\cdot \mathbf E^{(j)\,\ast})\,E_{s,i},\\
\text{(B)}:\quad & \chi_B^{(j)}\,\delta_{ik}\delta_{jl}\,E_{s,j}\,E^{(j)}_{k}E^{(j)\,\ast}_{l}
=\chi_B^{(j)}\,E^{(j)}_{i}\,(\mathbf E_s\!\cdot \mathbf E^{(j)\,\ast}),\\
\text{(C)}:\quad & \chi_C^{(j)}\,\delta_{il}\delta_{jk}\,E_{s,j}\,E^{(j)}_{k}E^{(j)\,\ast}_{l}
=\chi_C^{(j)}\,E^{(j)\,\ast}_{i}\,(\mathbf E_s\!\cdot \mathbf E^{(j)}).
\end{align}
Summing $j=1,2$,
\begin{equation}\label{eq:P3_operator}
P^{(3)}_i(\omega_s)=\varepsilon_0\sum_{m=1}^2
\Big[
\chi_A^{(j)}(\mathbf E^{(j)}\!\cdot \mathbf E^{(j)\,\ast})\,E_{s,i}
+\chi_B^{(j)}\,E^{(j)}_{i}\,(\mathbf E_s\!\cdot \mathbf E^{(j)\,\ast})
+\chi_C^{(j)}\,E^{(j)\,\ast}_{i}\,(\mathbf E_s\!\cdot \mathbf E^{(j)})
\Big].
\end{equation}
Equation~\eqref{eq:P3_operator} is linear in $\mathbf E_s$; thus we identify the \emph{effective} rank-2 tensor
\begin{equation}\label{eq:chi_eff_tensor}
\boxed{~
\chi^{\mathrm{eff}}_{ij}(\omega_s)=\sum_{m=1}^2\Big[
\chi_A^{(m)}(\mathbf E^{(m)}\!\cdot \mathbf E^{(m)\,\ast})\,\delta_{ij}
+\chi_B^{(m)}\,E^{(m)}_{i}\,E^{(m)\,\ast}_{j}
+\chi_C^{(m)}\,E^{(m)\,\ast}_{i}\,E^{(m)}_{j}
\Big],
~}
\end{equation}
such that $P^{(3)}_i=\varepsilon_0\,\chi^{\mathrm{eff}}_{ij}E_{s,j}$. All fast phase factors are contained in the frequency labels of $\chi^{(3)}$ and the phasors; no residual time dependence remains at $\omega_s$ because of the $\delta(\omega_a-\omega_b)$ selection.

\section{Specialization to Circular Pumps and Circular Eigenbasis}
Let the beams propagate along $+\hat{\mathbf z}$. With
$\mathbf E^{(1)}=E^{(1)}_+\,\hat{\mathbf e}_+$ and
$\mathbf E^{(2)}=E^{(2)}_-\,\hat{\mathbf e}_-$, the scalar products are
$\mathbf E^{(1)}\!\cdot \mathbf E^{(1)\,\ast}=|E^{(1)}_+|^2$ and
$\mathbf E^{(2)}\!\cdot \mathbf E^{(2)\,\ast}=|E^{(2)}_-|^2$.
The dyadics are
$E^{(1)}_{i}E^{(1)\,\ast}_{j}=|E^{(1)}_+|^2\,(\hat e_+)_i(\hat e_+)^\ast_j$ and
$E^{(2)}_{i}E^{(2)\,\ast}_{j}=|E^{(2)}_-|^2\,(\hat e_-)_i(\hat e_-)^\ast_j$,
and similarly with $i\leftrightarrow j$.
In the circular basis $\{\hat{\mathbf e}_+,\hat{\mathbf e}_-\}$, $\underline{\underline{\chi}}^{\,\mathrm{eff}}(\omega_s)$ is diagonal at the carrier:
\begin{equation}\label{eq:chi_eff_diag}
\boxed{~
\begin{aligned}
\chi^{\mathrm{eff}}_{++}&=\sum_{m=1}^2\chi_A^{(j)}|E^{(j)}|^2+(\chi_B^{(1)}+\chi_C^{(1)})|E^{(1)}_+|^2,\\
\chi^{\mathrm{eff}}_{--}&=\sum_{m=1}^2\chi_A^{(j)}|E^{(j)}|^2+(\chi_B^{(2)}+\chi_C^{(2)})|E^{(2)}_-|^2,\\
\chi^{\mathrm{eff}}_{+-}&=\chi^{\mathrm{eff}}_{-+}=0.
\end{aligned}
~}
\end{equation}
It is convenient to regroup (per tone) into scalar and axial combinations
\begin{equation}\label{eq:iso_circ_defs}
\chi_{\mathrm{iso}}^{(j)}\equiv \chi_A^{(j)}+\frac{\chi_B^{(j)}+\chi_C^{(j)}}{2},
\qquad
\chi_{\mathrm{circ}}^{(j)}\equiv \frac{\chi_B^{(j)}+\chi_C^{(j)}}{2},
\end{equation}
and to define tone-summed pump invariants at the carrier
\begin{equation}\label{eq:Itot_S3}
I_{\mathrm{tot}}\equiv |E^{(1)}_+|^2+|E^{(2)}_-|^2,
\qquad
S_3^{\mathrm{DC}}\equiv |E^{(1)}_+|^2-|E^{(2)}_-|^2\ \propto\ i(\mathbf E_p\times\mathbf E_p^\ast)\!\cdot\hat{\mathbf z}.
\end{equation}
Equation~\eqref{eq:chi_eff_diag} becomes
\begin{equation}\label{eq:chi_eff_final}
\boxed{~
\chi^{\mathrm{eff}}_{\pm\pm}(\omega_s)
=\Big[\sum_{m=1}^2 \chi_{\mathrm{iso}}^{(j)}\Big]\ I_{\mathrm{tot}}
\ \pm\ 
\Big[\sum_{m=1}^2 \chi_{\mathrm{circ}}^{(j)}\Big]\ S_3^{\mathrm{DC}}.
~}
\end{equation}
The first term is a true scalar (Kerr/XPM), while the second is an axial pseudoscalar (helicity or inverse Faraday effect), which splits the circular eigenmodes ($\sigma^\pm$), i.e.\ produces circular birefringence.

% ----------------------------------------------------------------
% Same-helicity pumps (σ+σ+ and σ−σ−): explicit tensor substitution
% ----------------------------------------------------------------
\subsection*{Same-helicity pumps (\texorpdfstring{$\sigma{+}\sigma{+}$}{sigma+ sigma+} and \texorpdfstring{$\sigma{-}\sigma{-}$}{sigma- sigma-})}

We substitute the pump fields directly into the effective third-order tensor
\begin{equation}
\chi^{\mathrm{eff}}_{ij}(\omega_s)
=\sum_{m=1}^2\!\left[
\chi_A^{(m)}\!\left(\mathbf E^{(m)}\!\cdot\!\mathbf E^{(m)\!*}\right)\delta_{ij}
+\chi_B^{(m)} E^{(m)}_{i} E^{(m)\!*}_{j}
+\chi_C^{(m)} E^{(m)\!*}_{i} E^{(m)}_{j}
\right],
\label{eq:chi-eff-samehelicity-start}
\end{equation}
which follows from contracting the isotropic rank-4 tensor
$\chi^{(3)}_{ijkl}=\chi_A\,\delta_{ij}\delta_{kl}+\chi_B\,\delta_{ik}\delta_{jl}+\chi_C\,\delta_{il}\delta_{jk}$
with one signal factor and \emph{same-tone} pump--pump factors at the signal carrier $\omega_s$.
Here $\chi_{A,B,C}^{(m)}\equiv\chi_{A,B,C}(-\omega_s;\omega_s,\omega_{p m},-\omega_{p m})$ and the tone
frequencies are $\omega_{p1}=\omega_p+\Delta/2$, $\omega_{p2}=\omega_p-\Delta/2$.
Only terms with $\omega_a=\omega_b$ contribute at $\omega_s$; cross-tone products generate sidebands at $\omega_s\pm\Delta$ and are excluded from $\chi^{\mathrm{eff}}(\omega_s)$.

\paragraph{Circular basis.}
Let $\hat{\mathbf e}_\pm=(\hat{\mathbf x}\mp i\hat{\mathbf y})/\sqrt{2}$,
$\hat{\mathbf e}_\pm\!\cdot\!\hat{\mathbf e}_\pm=1$, $\hat{\mathbf e}_+\!\cdot\!\hat{\mathbf e}_-=0$,
and note $\hat{\mathbf e}_+^{\,*}=\hat{\mathbf e}_-$, $\hat{\mathbf e}_-^{\,*}=\hat{\mathbf e}_+$.

% ------------------------------
% Case A: sigma+ sigma+
% ------------------------------
\subsubsection*{Case A: $\sigma{+}\sigma{+}$ (both pumps right-circular)}
Pumps:
\begin{equation}
\mathbf E^{(1)}=E^{(1)}_{+}\,\hat{\mathbf e}_+,\qquad
\mathbf E^{(2)}=E^{(2)}_{+}\,\hat{\mathbf e}_+ .
\end{equation}
Per tone,
\begin{equation}
\mathbf E^{(m)}\!\cdot\!\mathbf E^{(m)\!*}=|E^{(m)}_{+}|^2,\quad
E^{(m)}_{i}E^{(m)\!*}_{j}=|E^{(m)}_{+}|^2\,(\hat e_{+})_i(\hat e_{-})_j,\quad
E^{(m)\!*}_{i}E^{(m)}_{j}=|E^{(m)}_{+}|^2\,(\hat e_{-})_i(\hat e_{+})_j .
\end{equation}
Insert into \eqref{eq:chi-eff-samehelicity-start} and sum $m=1,2$:
\begin{equation}
\chi^{\mathrm{eff}}_{ij}(\omega_s)=
\underbrace{\Big(\sum_{m}\chi_A^{(m)}|E^{(m)}_{+}|^2\Big)}_{\displaystyle \Xi_A^{(+)}}
\delta_{ij}
+\underbrace{\Big(\sum_{m}\chi_B^{(m)}|E^{(m)}_{+}|^2\Big)}_{\displaystyle \Xi_B^{(+)}}
(\hat e_{+})_i(\hat e_{-})_j
+\underbrace{\Big(\sum_{m}\chi_C^{(m)}|E^{(m)}_{+}|^2\Big)}_{\displaystyle \Xi_C^{(+)}}
(\hat e_{-})_i(\hat e_{+})_j .
\end{equation}
In the circular basis $\{\hat{\mathbf e}_+,\hat{\mathbf e}_-\}$ the matrix representation is
\begin{equation}
\big[\chi^{\mathrm{eff}}(\omega_s)\big]_{\mathrm{(circ)}}^{(++)}=
\begin{pmatrix}
\Xi_{++}^{(+)} & 0\\[2pt]
0 & \Xi_{--}^{(+)}
\end{pmatrix},,
\qquad
\Xi_{A,B,C}^{(+)}=\sum_{m=1}^{2}\chi_{A,B,C}^{(m)}\,|E_{+}^{(m)}|^2 .
\label{eq:chi-eff-matrix-plusplus}
\end{equation}

Here the diagonal entries in the circular basis are
\begin{equation}
\Xi_{++}^{(+)}=\Xi_A^{(+)}+\Xi_B^{(+)}+\Xi_C^{(+)},\qquad
\Xi_{--}^{(+)}=\Xi_A^{(+)}-(\Xi_B^{(+)}+\Xi_C^{(+)}).
\end{equation}
% ------------------------------
% Case B: sigma- sigma-
% ------------------------------
\subsubsection*{Case B: $\sigma{-}\sigma{-}$ (both pumps left-circular)}
Pumps:
\begin{equation}
\mathbf E^{(1)}=E^{(1)}_{-}\,\hat{\mathbf e}_-,\qquad
\mathbf E^{(2)}=E^{(2)}_{-}\,\hat{\mathbf e}_- .
\end{equation}
Per tone,
\begin{equation}
\mathbf E^{(m)}\!\cdot\!\mathbf E^{(m)\!*}=|E^{(m)}_{-}|^2,\quad
E^{(m)}_{i}E^{(m)\!*}_{j}=|E^{(m)}_{-}|^2\,(\hat e_{-})_i(\hat e_{+})_j,\quad
E^{(m)\!*}_{i}E^{(m)}_{j}=|E^{(m)}_{-}|^2\,(\hat e_{+})_i(\hat e_{-})_j .
\end{equation}
Insert into \eqref{eq:chi-eff-samehelicity-start} and sum $m=1,2$:
\begin{equation}
\chi^{\mathrm{eff}}_{ij}(\omega_s)=
\underbrace{\Big(\sum_{m}\chi_A^{(m)}|E^{(m)}_{-}|^2\Big)}_{\displaystyle \Xi_A^{(-)}}
\delta_{ij}
+\underbrace{\Big(\sum_{m}\chi_B^{(m)}|E^{(m)}_{-}|^2\Big)}_{\displaystyle \Xi_B^{(-)}}
(\hat e_{-})_i(\hat e_{+})_j
+\underbrace{\Big(\sum_{m}\chi_C^{(m)}|E^{(m)}_{-}|^2\Big)}_{\displaystyle \Xi_C^{(-)}}
(\hat e_{+})_i(\hat e_{-})_j .
\end{equation}
In the circular basis $\{\hat{\mathbf e}_+,\hat{\mathbf e}_-\}$ the matrix representation is
\begin{equation}
\big[\chi^{\mathrm{eff}}(\omega_s)\big]_{\mathrm{(circ)}}^{(--)}=
\begin{pmatrix}
\Xi_{++}^{(-)} & 0\\[2pt]
0 & \Xi_{--}^{(-)}
\end{pmatrix},,
\qquad
\Xi_{A,B,C}^{(-)}=\sum_{m=1}^{2}\chi_{A,B,C}^{(m)}\,|E_{-}^{(m)}|^2 .
\label{eq:chi-eff-matrix-minusminus}
\end{equation}

Analogously,
\begin{equation}
\Xi_{++}^{(-)}=\Xi_A^{(-)}+\Xi_B^{(-)}+\Xi_C^{(-)},\qquad
\Xi_{--}^{(-)}=\Xi_A^{(-)}-(\Xi_B^{(-)}+\Xi_C^{(-)}).
\end{equation}

\paragraph{Remarks.}
(i) In an axially symmetric (same-helicity) pump configuration the circular basis diagonalizes the tensor, so $\chi^{\mathrm{eff}}_{+-}=\chi^{\mathrm{eff}}_{-+}=0$ at the carrier; off-diagonal terms vanish.
(ii) Any axial, time-reversal-odd (Faraday-active) contribution appears as a \emph{difference of circular eigenvalues}
in the circular basis; it is tied to the pump helicity density $i\,\mathbf E\times\mathbf E^{\!*}$ and produces
circular birefringence $(n_+\neq n_-)$ and a carrier Faraday rotation $\theta_F=\tfrac{k_0 L}{2}(n_+-n_-)$.

\paragraph{Units and intensity conversion.}
We use $|E|^2 = 2I/(n\,\varepsilon_0 c)$ to connect intensities to field amplitudes. Accordingly, the Kerr index coefficients obey $n_{2,\cdot} = \tfrac{3}{4 n_0 n_p\,\varepsilon_0 c}\,\mathrm{Re}[\chi_{\cdot}^{(3)}]$ under isotropy (Kleinman), with $n_0$ the probe index and $n_p$ the pump index at their respective frequencies.


\section{Eigenindices and Faraday Rotation}
Treat the pump-dressed probe response as effectively linear at $\omega_s$:
\begin{equation}
D_i(\omega_s)=\varepsilon_0\Big[\varepsilon^{(1)}_{ij}(\omega_s)+\chi^{\mathrm{eff}}_{ij}(\omega_s)\Big]E_{s,j}(\omega_s).
\end{equation}
For small nonlinearity,
\begin{equation}\label{eq:n_linearize}
n_\pm^2(\omega_s)\simeq n_0^2(\omega_s)+\mathrm{Re}\,\chi^{\mathrm{eff}}_{\pm\pm}(\omega_s)
\quad\Rightarrow\quad
\delta n_\pm\simeq \frac{\mathrm{Re}\,\chi^{\mathrm{eff}}_{\pm\pm}}{2\,n_0}.
\end{equation}
Converting field to intensity via $I=(n_0\varepsilon_0 c/2)|E|^2$, it is convenient to define Kerr coefficients (SI units)
\begin{equation}\label{eq:n2_def}
n_{2,\{\mathrm{iso},\mathrm{circ}\}}
=\frac{3}{4\,n_0\,n_0'\,\varepsilon_0 c}\,
\mathrm{Re}\Big[\sum_{j}\chi_{\{\mathrm{iso},\mathrm{circ}\}}^{(j)}(-\omega_s;\omega_s,\omega_{p j},-\omega_{p j})\Big],
\end{equation}
so that
\begin{equation}\label{eq:npm_final}
\boxed{~
n_\pm(\omega_s)=n_0(\omega_s)\ +\ n_{2,\mathrm{iso}}\,I_{\mathrm{tot}}
\ \pm\ n_{2,\mathrm{circ}}\,S_3^{\mathrm{DC}}.
~}
\end{equation}
A linearly polarized probe (equal $\sigma^\pm$ superposition) accumulates a relative phase
\begin{equation}\label{eq:delta_phi}
\Delta\phi=k_0 L\,[n_+-n_-]=2k_0L\,n_{2,\mathrm{circ}}\,S_3^{\mathrm{DC}},\qquad k_0=\omega_s/c,
\end{equation}
so the Faraday rotation (carrier component) is
\begin{equation}\label{eq:thetaF}
\boxed{~
\theta_F^{\mathrm{(carrier)}}=\frac{\Delta\phi}{2}
=\frac{k_0 L}{2}\,[n_+-n_-]
= k_0 L\,n_{2,\mathrm{circ}}\,S_3^{\mathrm{DC}}.
~}
\end{equation}
If the two counter-helicity pumps have equal intensities, $S_3^{\mathrm{DC}}=0\Rightarrow \theta_F^{\mathrm{(carrier)}}=0$. Flipping which helicity is stronger flips the sign of $S_3^{\mathrm{DC}}$ and hence of $\theta_F$ (inverse Faraday effect sign).

\section{Cross-Tone (Beat) Terms and Sidebands}
Keeping cross-tone contractions in Eq.~\eqref{eq:P3_general} produces terms $\propto E^{(1)} E^{(2)\ast}e^{\mp i\Delta t}$ that contribute to $\mathbf P^{(3)}(\omega_s\pm\Delta)$. Physically, these modulate the probe phase/polarization at $\Delta$ and generate sidebands; they are \emph{excluded} from the carrier effective tensor by the selection rule $\delta(\omega_a-\omega_b)$ in Eq.~\eqref{eq:P3_signal}. Time-resolved polarimetry with bandwidth $\gtrsim \Delta$ can observe an oscillatory rotation, but its spectral home is the sidebands, not $\omega_s$.

% ===============================================================
% Sideband-dressed effective susceptibility at the carrier \omega_s
% (uses the same notation and conventions as this file)
% ===============================================================
\section{Sideband-dressed $\underline{\underline{\chi}}^{\,\mathrm{eff}}(\omega_s)$ and Faraday rotation}

\paragraph{Set-up and selection rules.}
We consider two strong tones at $\omega_{p\pm}=\omega_p\pm\Delta/2$ (Sec.~1)
and a weak probe at $\omega_s$. Cross-tone contractions of $\chi^{(3)}_{ijkl}$
with the pumps generate probe \emph{sidebands} at $\omega_s\pm\Delta$, cf.\ Sec.~``Cross-Tone (Beat) Terms.''
Retaining only the first sidebands and working to linear order in the probe,
the probe spectral amplitudes $E_s(\omega_s)$, $E_{s,\pm}\equiv E_s(\omega_s\pm\Delta)$ obey a $3\times 3$
coupled system. Eliminating $E_{s,\pm}$ yields a \emph{sideband-dressed} effective tensor
at the carrier $\omega_s$. Below we give the explicit algebra in our isotropic-tensor notation.

\paragraph{Coupled system and elimination.}
In frequency space the wave equation can be written schematically as
$\mathcal{D}(\omega)\,E_i(\omega)=(\omega^2/\varepsilon_0 c^2)\,P^{(3)}_i(\omega)$.
Near $\omega_s$ we define lumped complex detunings
\begin{equation}
\mathcal D_0\equiv \mathcal D(\omega_s),\qquad
\mathcal D_\pm\equiv \mathcal D(\omega_s\pm\Delta)=\Gamma_s\mp i\Delta,
\end{equation}
where $\Gamma_s>0$ is the probe's linear loss (phenomenological).

Write the cross-tone third-order polarization pieces that connect the probe
to its sidebands in the compact \emph{tensor-contracted} form (keeping only terms linear in $\mathbf E_s$):
\begin{align}
P^{(3)}_i(\omega_s) &\supset \varepsilon_0\,
\Big[\,\mathcal M^{(-)}_{ij}\,E_{s,-,j}+\mathcal M^{(+)}_{ij}\,E_{s,+,j}\,\Big],\\
P^{(3)}_i(\omega_s\pm\Delta) &\supset \varepsilon_0\,
\Big[\,\widetilde{\mathcal M}^{(\pm)}_{ij}\,E_{s,j}\,\Big],
\end{align}
with the \emph{mixing tensors} (from the isotropic decomposition \eqref{eq:isotropic_decomp})
\begin{align}
\mathcal M^{(\pm)}_{ij}
&=\sum_{m=A,B,C}\!\chi^{(m)}_{(\mp 1)}(\omega_s;\omega_{p+},-\omega_{p-},\omega_s\pm\Delta)\;\,
\mathcal T^{(m)}_{ij}\big[\mathbf E^{(1)},\mathbf E^{(2)}\big],\\
\widetilde{\mathcal M}^{(\pm)}_{ij}
&=\sum_{m=A,B,C}\!\chi^{(m)}_{(\pm 1)}(\omega_s\pm\Delta;\omega_{p+},-\omega_{p-},\omega_s)\;\,
\mathcal T^{(m)}_{ij}\big[\mathbf E^{(1)},\mathbf E^{(2)}\big],
\label{eq:M-mixers}
\end{align}
where $\chi^{(m)}_{(\pm1)}$ denote the \emph{Floquet blocks} of the third-order kernel indexed by the modulation harmonic $\pm 1$,
and $\mathcal T^{(A,B,C)}_{ij}$ are the contractions produced by the three isotropic invariants:
\begin{equation}
\mathcal T^{(A)}_{ij}=(\mathbf E^{(1)}\!\cdot\mathbf E^{(2)\!*})\,\delta_{ij},\quad
\mathcal T^{(B)}_{ij}=E^{(1)}_{i}E^{(2)\!*}_{j},\quad
\mathcal T^{(C)}_{ij}=E^{(1)\!*}_{i}E^{(2)}_{j}.
\end{equation}

Collect the three probe unknowns as $\mathcal E\equiv\big(E_{s,i},E_{s,+,i},E_{s,-,i}\big)^{\mathrm T}$.
To first order one obtains
\begin{equation}
\begin{pmatrix}
\mathcal D_0\,\delta_{ij} & -\mathcal K^{(+)}_{ij} & -\mathcal K^{(-)}_{ij} \\[2pt]
-\widetilde{\mathcal K}^{(+)}_{ij} & \mathcal D_{+}\,\delta_{ij} & 0 \\[2pt]
-\widetilde{\mathcal K}^{(-)}_{ij} & 0 & \mathcal D_{-}\,\delta_{ij}
\end{pmatrix}
\!
\begin{pmatrix}
E_{s,j}\\ E_{s,+,j}\\ E_{s,-,j}
\end{pmatrix}
=
\begin{pmatrix}
S_{s,i}\\ 0\\ 0
\end{pmatrix},
\qquad
\mathcal K^{(\pm)}_{ij}\equiv \frac{\omega_s^2}{c^2}\,\mathcal M^{(\pm)}_{ij},\quad
\widetilde{\mathcal K}^{(\pm)}_{ij}\equiv \frac{(\omega_s\pm\Delta)^2}{c^2}\,\widetilde{\mathcal M}^{(\pm)}_{ij}.
\label{eq:3by3-system}
\end{equation}
Here $S_{s,i}$ is the external probe drive at $\omega_s$ (port/boundary excitation).
Assuming $||\mathcal K^{(\pm)}||,||\widetilde{\mathcal K}^{(\pm)}||\ll |\mathcal D_\pm|$ (weak sideband excitation),
solve the lower rows for the sidebands and substitute back:
\begin{equation}
E_{s,\pm,i}\approx \big(\mathcal D_\pm^{-1}\,\widetilde{\mathcal K}^{(\pm)}\big)_{ij}E_{s,j}.
\end{equation}
The resulting \emph{sideband-dressed} constitutive relation at $\omega_s$ is
\begin{equation}
P^{(3)}_i(\omega_s)=\varepsilon_0\Big[\chi^{\mathrm{eff}}_{ij}(\omega_s)
+\delta\chi^{(+)}_{ij}(\omega_s)+\delta\chi^{(-)}_{ij}(\omega_s)\Big]E_{s,j},
\end{equation}
with the corrections
\begin{equation}
\boxed{~
\delta\chi^{(\pm)}_{ij}(\omega_s)
=\frac{c^2}{\omega_s^2}\,
\big(\mathcal K^{(\pm)}\,\mathcal D_\pm^{-1}\,\widetilde{\mathcal K}^{(\pm)}\big)_{ij}
=\frac{1}{\mathcal D_\pm}\,
\mathcal M^{(\pm)}_{ik}\,\widetilde{\mathcal M}^{(\pm)}_{kj}
\quad\propto\quad \frac{|E^{(1)}|^2|E^{(2)}|^2}{\Gamma_s\mp i\Delta}\,.
~}
\label{eq:dchi-final}
\end{equation}
Equations \eqref{eq:M-mixers}--\eqref{eq:dchi-final} are the sought \emph{same-notation} expression for the
sideband-dressed effective tensor at the carrier.

\paragraph{Circular basis: diagonal elements.}
Projecting \eqref{eq:dchi-final} onto the circular eigenbasis $\{\hat{\mathbf e}_\pm\}$
and assuming collinear propagation (our Sec.~``Specialization to Circular Pumps''), the effective
diagonal elements become
\begin{equation}
\boxed{~
\chi^{\mathrm{eff,SB}}_{\pm\pm}(\omega_s)
=\chi^{\mathrm{eff}}_{\pm\pm}(\omega_s)
+\big[\delta\chi^{(+)}_{\pm\pm}(\omega_s)+\delta\chi^{(-)}_{\pm\pm}(\omega_s)\big],
\qquad
\delta\chi^{(\pm)}_{\alpha\alpha}=\hat e_{\alpha,i}\,\delta\chi^{(\pm)}_{ij}\,\hat e_{\alpha,j}^{\!*}\ \ (\alpha=\pm).
~}
\label{eq:chi-eff-SB-diag}
\end{equation}

\paragraph{Cases: $\sigma^+\sigma^-$ and $\sigma^+\sigma^+$ pumps.}
Let $\mathbf E^{(1)}=E^{(1)}_{+}\hat{\mathbf e}_+$ and $\mathbf E^{(2)}=E^{(2)}_{-}\hat{\mathbf e}_-$ for the
$\sigma^+\sigma^-$ case, and $\mathbf E^{(1)}=E^{(1)}_{+}\hat{\mathbf e}_+$,
$\mathbf E^{(2)}=E^{(2)}_{+}\hat{\mathbf e}_+$ for the $\sigma^+\sigma^+$ case (the $\sigma^-\sigma^-$
case follows by $+\leftrightarrow-$). Using the isotropic contractions:
\begin{align}
\sigma^+\sigma^-:\qquad
&\mathcal T^{(A)}_{ij}=E^{(1)}_{+}E^{(2)}_{-}\,\delta_{ij},\ \
\mathcal T^{(B)}_{ij}=E^{(1)}_{+}E^{(2)}_{-}\,(\hat e_+)_{i}(\hat e_-)_{j},\ \
\mathcal T^{(C)}_{ij}=E^{(1)}_{+}E^{(2)}_{-}\,(\hat e_-)_{i}(\hat e_+)_{j},\\
\sigma^+\sigma^+:\qquad
&\mathcal T^{(A)}_{ij}=E^{(1)}_{+}E^{(2)}_{+}\,\delta_{ij},\ \
\mathcal T^{(B)}_{ij}=E^{(1)}_{+}E^{(2)}_{+}\,(\hat e_+)_{i}(\hat e_+)_{j},\ \
\mathcal T^{(C)}_{ij}=E^{(1)}_{+}E^{(2)}_{+}\,(\hat e_-)_{i}(\hat e_-)_{j}.
\end{align}
Substituting into \eqref{eq:M-mixers}–\eqref{eq:chi-eff-SB-diag} gives compact diagonal results:
\begin{align}
\boxed{~
\sigma^+\sigma^-:\quad
\begin{aligned}
\delta\chi^{(\pm)}_{++}&=\frac{|E^{(1)}_{+}E^{(2)}_{-}|^2}{\mathcal D_\pm}
\Big[\chi^{(+)}_{(\mp1),B}+\chi^{(+)}_{(\mp1),C}\Big]\Big[\chi^{(+)}_{(\pm1),B}+\chi^{(+)}_{(\pm1),C}\Big],\\
\delta\chi^{(\pm)}_{--}&=\frac{|E^{(1)}_{+}E^{(2)}_{-}|^2}{\mathcal D_\pm}
\Big[\chi^{(-)}_{(\mp1),B}+\chi^{(-)}_{(\mp1),C}\Big]\Big[\chi^{(-)}_{(\pm1),B}+\chi^{(-)}_{(\pm1),C}\Big],
\end{aligned}
~}
\label{eq:SB-pm}
\\[4pt]
\boxed{~
\sigma^+\sigma^+:\quad
\begin{aligned}
\delta\chi^{(\pm)}_{++}&=\frac{|E^{(1)}_{+}E^{(2)}_{+}|^2}{\mathcal D_\pm}
\Big[\chi^{(+)}_{(\mp1),A}+\chi^{(+)}_{(\mp1),B}+\chi^{(+)}_{(\mp1),C}\Big]
\Big[\chi^{(+)}_{(\pm1),A}+\chi^{(+)}_{(\pm1),B}+\chi^{(+)}_{(\pm1),C}\Big],\\
\delta\chi^{(\pm)}_{--}&=\frac{|E^{(1)}_{+}E^{(2)}_{+}|^2}{\mathcal D_\pm}
\Big[\chi^{(-)}_{(\mp1),A}-(\chi^{(-)}_{(\mp1),B}+\chi^{(-)}_{(\mp1),C})\Big]
\Big[\chi^{(-)}_{(\pm1),A}-(\chi^{(-)}_{(\pm1),B}+\chi^{(-)}_{(\pm1),C})\Big].
\end{aligned}
~}
\label{eq:SB-pp}
\end{align}
Here $\chi^{(\pm)}_{(\cdot),m}$ are the scalar amplitudes multiplying the isotropic $A,B,C$ contractions for the
Floquet-harmonic blocks that couple the indicated circular component. (The precise microscopic values
follow from the density-matrix calculation used elsewhere in this file.)

\paragraph{Faraday rotation at the carrier.}
For small nonlinearities, $n_\pm(\omega_s)\simeq 1+\tfrac{1}{2}\mathrm{Re}\,\chi_{\pm\pm}(\omega_s)$.
A linearly polarized probe thus acquires the rotation
\begin{equation}
\boxed{~
\theta_F(\omega_s)=\frac{k_0 L}{4}\,
\mathrm{Re}\!\left\{\big[\chi^{\mathrm{eff}}_{++}-\chi^{\mathrm{eff}}_{--}\big]
+\big[\delta\chi^{(+)}_{++}-\delta\chi^{(+)}_{--}\big]
+\big[\delta\chi^{(-)}_{++}-\delta\chi^{(-)}_{--}\big]\right\}_{\omega_s}.
~}
\label{eq:thetaF-final}
\end{equation}
\emph{Discussion.} (i) For \underline{$\sigma^+\sigma^-$ pumps} with equal intensities and an isotropic, parity-symmetric medium,
the stationary DC contribution satisfies $\chi^{\mathrm{eff}}_{++}=\chi^{\mathrm{eff}}_{--}$ (Sec.~``Specialization to Circular Pumps''),
so $\theta_F$ stems solely from the sideband loops in \eqref{eq:SB-pm} and is
\emph{dispersive}, $\propto \mathrm{Re}(1/\mathcal D_\pm)$; it vanishes for $\Delta\to\infty$ and is maximal near $|\Delta|\sim\Gamma_s$.\\
(ii) For \underline{$\sigma^+\sigma^+$ pumps} (or $\sigma^-\sigma^-$) the stationary term already yields
$\chi^{\mathrm{eff}}_{++}\neq \chi^{\mathrm{eff}}_{--}$ (Sec.~``Same-helicity pumps''), producing a nonzero $\theta_F$ even without sidebands.
Equations \eqref{eq:SB-pp}–\eqref{eq:thetaF-final} quantify the additional sideband dressing; the sign of
$\mathrm{Re}\,[\delta\chi_{++}-\delta\chi_{--}]$ flips with $\Delta$ and with pump helicity.

\medskip
The expressions above are fully consistent with the isotropic-tensor contraction in Eq.~\eqref{eq:isotropic_decomp} and the
circular-basis diagonalization in Eq.~\eqref{eq:chi_eff_final}, while incorporating the cross-tone sidebands through
the elimination procedure \eqref{eq:3by3-system}–\eqref{eq:dchi-final}.


% ======================================================================
% Section: Sideband-dressed chi^(3)_eff and time-resolved Faraday rotation
% ======================================================================
\section{Time-resolved Faraday rotation}

\subsection{Fields, basis, and isotropic contractions}

\paragraph{Probe (weak).}
A monochromatic probe at $\omega_s$ with Jones vector $\mathbf E_s$:
\[
\mathbf E_s(t)=\tfrac{1}{2}\!\left[\mathbf E_s\,e^{-i\omega_s t}+\mathrm{c.c.}\right],\qquad
\|\mathbf E_s\|\ll \|\mathbf E^{(1,2)}\|.
\]

\paragraph{Pumps (two strong tones, beat at $\Delta$).}
Two pumps at
\[
\omega_{p\pm}=\omega_p\pm\Delta/2,\qquad
\mathbf E^{(1)}(t)=\tfrac12\!\left[\mathbf E^{(1)}e^{-i\omega_{p+}t}+\mathrm{c.c.}\right],\quad
\mathbf E^{(2)}(t)=\tfrac12\!\left[\mathbf E^{(2)}e^{-i\omega_{p-}t}+\mathrm{c.c.}\right].
\]
Their product carries a time modulation at $\Delta$, which couples $\omega_s\leftrightarrow \omega_s\pm\Delta$ (Floquet picture).

\paragraph{Circular basis.}
We use $\{\hat{\mathbf e}_+,\hat{\mathbf e}_-\}$ for propagation along $+\hat z$, with
$\mathbf E_s=E_{s,+}\hat{\mathbf e}_++E_{s,-}\hat{\mathbf e}_-$.

% \paragraph{Isotropic $\chi^{(3)}$ contractions (your $A,B,C$).}
% Let $\mathcal T^{(A,B,C)}_{ij}$ denote the three isotropic rank-2 contractions built from the pump vectors:
% \[
% \mathcal T^{(A)}_{ij}=(\mathbf E^{(1)}\!\cdot\mathbf E^{(2)\!*})\,\delta_{ij},\quad
% \mathcal T^{(B)}_{ij}=E^{(1)}_{i}\,E^{(2)\!*}_{j},\quad
% \mathcal T^{(C)}_{ij}=E^{(1)\!*}_{i}\,E^{(2)}_{j}.
% \]
% We denote the (Floquet) third-order scalars that multiply these as $\chi^{(m)}_{(0)}$ (stationary), and
% $\chi^{(m)}_{(\pm1)}$ (first sideband blocks), with $m\in\{A,B,C\}$.
%
% \subsection{Frequency-domain sideband elimination and $\delta\underline{\underline{\chi}}(\omega_s)$}
%
% Retaining only terms linear in $\mathbf E_s$ and quadratic in the pumps, the third-order polarization pieces that mix the probe carrier with its first sidebands are
% \begin{align}
% P^{(3)}_i(\omega_s) &\supset \varepsilon_0\left[\mathcal M^{(-)}_{ij}\,E_{s,-,j}+\mathcal M^{(+)}_{ij}\,E_{s,+,j}\right],\\
% P^{(3)}_i(\omega_s\pm\Delta) &\supset \varepsilon_0\,\widetilde{\mathcal M}^{(\pm)}_{ij}\,E_{s,j},
% \end{align}
% with \emph{mixing tensors}
% \begin{equation}
% \boxed{~
% \mathcal M^{(\pm)}_{ij}=\sum_{m=A,B,C}\chi^{(m)}_{(\mp 1)}\!\big(\omega_s;\omega_{p+},-\omega_{p-},\omega_s\pm\Delta\big)\,\mathcal T^{(m)}_{ij},\qquad
% \widetilde{\mathcal M}^{(\pm)}_{ij}=\sum_{m=A,B,C}\chi^{(m)}_{(\pm 1)}\!\big(\omega_s\pm\Delta;\omega_{p+},-\omega_{p-},\omega_s\big)\,\mathcal T^{(m)}_{ij}.~}
% \label{eq:mixers}
% \end{equation}
%
% Introduce the linear propagation operator $\mathcal D(\omega)$ and its lumped values at the three frequencies,
% \[
% \mathcal D_0\equiv \mathcal D(\omega_s),\qquad
% \mathcal D_\pm\equiv \mathcal D(\omega_s\pm\Delta)=\Gamma_s\mp i\Delta,
% \]
% where $\Gamma_s>0$ is the probe’s linear decoherence/linewidth near $\omega_s$.
%
% Collecting the unknown probe amplitudes $(E_{s,i},E_{s,+,i},E_{s,-,i})$, the first-order coupled system reads
% \begin{equation}
% \begin{pmatrix}
% \mathcal D_0\,\delta_{ij} & -\mathcal K^{(+)}_{ij} & -\mathcal K^{(-)}_{ij} \\
% -\widetilde{\mathcal K}^{(+)}_{ij} & \mathcal D_{+}\,\delta_{ij} & 0 \\
% -\widetilde{\mathcal K}^{(-)}_{ij} & 0 & \mathcal D_{-}\,\delta_{ij}
% \end{pmatrix}
% \!
% \begin{pmatrix}
% E_{s,j}\\ E_{s,+,j}\\ E_{s,-,j}
% \end{pmatrix}
% =
% \begin{pmatrix}
% S_{s,i}\\ 0\\ 0
% \end{pmatrix},
% \qquad
% \mathcal K^{(\pm)}_{ij}\equiv \frac{\omega_s^2}{c^2}\,\mathcal M^{(\pm)}_{ij},\ \
% \widetilde{\mathcal K}^{(\pm)}_{ij}\equiv \frac{(\omega_s\pm\Delta)^2}{c^2}\,\widetilde{\mathcal M}^{(\pm)}_{ij}.
% \label{eq:3x3}
% \end{equation}
% Here $S_{s,i}$ is the external probe drive at $\omega_s$ (port/boundary excitation).
%
% In the weak-mixing limit $\|\mathcal K^{(\pm)}\|,\|\widetilde{\mathcal K}^{(\pm)}\|\ll|\mathcal D_\pm|$,
% \[
% E_{s,\pm,i}\approx (\mathcal D_\pm^{-1}\,\widetilde{\mathcal K}^{(\pm)})_{ij}E_{s,j},
% \]
% and the carrier constitutive relation becomes
% \[
% P^{(3)}_i(\omega_s)=\varepsilon_0\left[\chi^{\mathrm{eff}}_{ij}(\omega_s)+\delta\chi^{(+)}_{ij}(\omega_s)+\delta\chi^{(-)}_{ij}(\omega_s)\right]E_{s,j},
% \]
% with the sideband corrections
% \begin{equation}
% \boxed{~
% \delta\chi^{(\pm)}_{ij}(\omega_s)
% =\frac{c^2}{\omega_s^2}\big(\mathcal K^{(\pm)}\,\mathcal D_\pm^{-1}\,\widetilde{\mathcal K}^{(\pm)}\big)_{ij}
% =\frac{1}{\mathcal D_\pm}\,\mathcal M^{(\pm)}_{ik}\,\widetilde{\mathcal M}^{(\pm)}_{kj}
% \ \ \propto\ \ \frac{|\,\mathbf E^{(1)}|^2|\,\mathbf E^{(2)}|^2}{\Gamma_s\mp i\Delta}\,.
% ~}
% \label{eq:dchi}
% \end{equation}
% Therefore
% \[
% \boxed{~
% \underline{\underline{\chi}}^{\,\mathrm{eff,SB}}(\omega_s)
% =\underline{\underline{\chi}}^{\,\mathrm{eff}}(\omega_s)+\delta\underline{\underline{\chi}}^{(+)}(\omega_s)+\delta\underline{\underline{\chi}}^{(-)}(\omega_s).~}
% \]
%
% \subsection{Circular-basis diagonals and explicit pump helicities}
%
% Projecting (\ref{eq:dchi}) to the circular basis gives
% \[
% \chi^{\mathrm{eff,SB}}_{\alpha\alpha}(\omega_s)
% =\chi^{\mathrm{eff}}_{\alpha\alpha}(\omega_s)+\sum_{\pm}\delta\chi^{(\pm)}_{\alpha\alpha}(\omega_s),\qquad
% \delta\chi^{(\pm)}_{\alpha\alpha}=\hat e_{\alpha,i}\,\delta\chi^{(\pm)}_{ij}\,\hat e_{\alpha,j}^{\!*},\ \ \alpha\in\{+,-\}.
% \]
%
% \paragraph{Case \texorpdfstring{$\sigma^+\sigma^-$}{sigma+ sigma-} pumps.}
% With $\mathbf E^{(1)}=E^{(1)}_+\hat{\mathbf e}_+$, $\mathbf E^{(2)}=E^{(2)}_-\hat{\mathbf e}_-$, the isotropic contractions yield
% \[
% \boxed{~
% \begin{aligned}
% \delta\chi^{(\pm)}_{++}&=\frac{|E^{(1)}_{+}E^{(2)}_{-}|^2}{\mathcal D_\pm}
% \Big[\chi^{(+)}_{(\mp1),B}+\chi^{(+)}_{(\mp1),C}\Big]\Big[\chi^{(+)}_{(\pm1),B}+\chi^{(+)}_{(\pm1),C}\Big],\\
% \delta\chi^{(\pm)}_{--}&=\frac{|E^{(1)}_{+}E^{(2)}_{-}|^2}{\mathcal D_\pm}
% \Big[\chi^{(-)}_{(\mp1),B}+\chi^{(-)}_{(\mp1),C}\Big]\Big[\chi^{(-)}_{(\pm1),B}+\chi^{(-)}_{(\pm1),C}\Big].
% \end{aligned}
% ~}
% \]
%
% \paragraph{Case \texorpdfstring{$\sigma^+\sigma^+$}{sigma+ sigma+} pumps.}
% With $\mathbf E^{(1)}=E^{(1)}_+\hat{\mathbf e}_+$, $\mathbf E^{(2)}=E^{(2)}_+\hat{\mathbf e}_+$:
% \[
% \boxed{~
% \begin{aligned}
% \delta\chi^{(\pm)}_{++}&=\frac{|E^{(1)}_{+}E^{(2)}_{+}|^2}{\mathcal D_\pm}
% \Big[\chi^{(+)}_{(\mp1),A}+\chi^{(+)}_{(\mp1),B}+\chi^{(+)}_{(\mp1),C}\Big]
% \Big[\chi^{(+)}_{(\pm1),A}+\chi^{(+)}_{(\pm1),B}+\chi^{(+)}_{(\pm1),C}\Big],\\
% \delta\chi^{(\pm)}_{--}&=\frac{|E^{(1)}_{+}E^{(2)}_{+}|^2}{\mathcal D_\pm}
% \Big[\chi^{(-)}_{(\mp1),A}-(\chi^{(-)}_{(\mp1),B}+\chi^{(-)}_{(\mp1),C})\Big]
% \Big[\chi^{(-)}_{(\pm1),A}-(\chi^{(-)}_{(\pm1),B}+\chi^{(-)}_{(\pm1),C})\Big].
% \end{aligned}
% ~}
% \]
%
\subsection{Temporal coupled-mode theory (TCMT) with sidebands and explicit $g_\pm,\kappa_\pm$}

Define slowly varying envelopes per circular component $\alpha\in\{+,-\}$:
$a_{0,\alpha}(t)$ at $\omega_s$, and $a_{\pm,\alpha}(t)$ at $\omega_s\pm\Delta$.
Let $\delta_{0,\alpha}$ and $\gamma_{0,\alpha}$ be the carrier detuning and decay,
and $\delta_\pm=\delta_0\pm\Delta$, $\gamma_\pm$ the sideband detuning/decay.
The input–output coupling is $\sqrt{2\gamma_{e,\alpha}}\,s_{\mathrm{in},\alpha}(t)$.

The TCMT equations are
\begin{equation}
\boxed{~
\begin{aligned}
\dot a_{0,\alpha}&=\big(-\gamma_{0,\alpha}-i\delta_{0,\alpha}\big)a_{0,\alpha}
+i\,\kappa_{+}^{(\alpha)}\,a_{+,\alpha}+i\,\kappa_{-}^{(\alpha)}\,a_{-,\alpha}
+\sqrt{2\gamma_{e,\alpha}}\,s_{\mathrm{in},\alpha}(t),\\
\dot a_{+,\alpha}&=\big(-\gamma_{+}-i\delta_{+}\big)a_{+,\alpha}+i\,g_{+}^{(\alpha)}\,a_{0,\alpha},\\
\dot a_{-,\alpha}&=\big(-\gamma_{-}-i\delta_{-}\big)a_{-,\alpha}+i\,g_{-}^{(\alpha)}\,a_{0,\alpha}.
\end{aligned}
~}
\label{eq:tcmt}
\end{equation}

The \emph{mixing coefficients} are circular projections of (\ref{eq:M-mixers}):
\[
\kappa_{\pm}^{(\alpha)}=\hat e_{\alpha,i}\,\mathcal M^{(\pm)}_{ij}\,\hat e_{\alpha,j}^{\!*},\qquad
g_{\pm}^{(\alpha)}=\hat e_{\alpha,i}\,\widetilde{\mathcal M}^{(\pm)}_{ij}\,\hat e_{\alpha,j}^{\!*}.
\]
Explicitly (list the nonzero scalars only):
\begin{align*}
\underline{\sigma^+\sigma^-:}\quad
&g_{+}^{(+)}=\big[\chi^{(+)}_{(+1),B}+\chi^{(+)}_{(+1),C}\big]E^{(1)}_+E^{(2)}_-,
&&\kappa_{+}^{(+)}=\big[\chi^{(+)}_{(-1),B}+\chi^{(+)}_{(-1),C}\big]E^{(1)}_+E^{(2)}_-,\\
&g_{-}^{(+)}=\big[\chi^{(+)}_{(-1),B}+\chi^{(+)}_{(-1),C}\big]E^{(1)}_+E^{(2)}_-,
&&\kappa_{-}^{(+)}=\big[\chi^{(+)}_{(+1),B}+\chi^{(+)}_{(+1),C}\big]E^{(1)}_+E^{(2)}_-,\\
&g_{+}^{(-)}=\big[\chi^{(-)}_{(+1),B}+\chi^{(-)}_{(+1),C}\big]E^{(1)}_+E^{(2)}_-,
&&\kappa_{+}^{(-)}=\big[\chi^{(-)}_{(-1),B}+\chi^{(-)}_{(-1),C}\big]E^{(1)}_+E^{(2)}_-,\\
&g_{-}^{(-)}=\big[\chi^{(-)}_{(-1),B}+\chi^{(-)}_{(-1),C}\big]E^{(1)}_+E^{(2)}_-,
&&\kappa_{-}^{(-)}=\big[\chi^{(-)}_{(+1),B}+\chi^{(-)}_{(+1),C}\big]E^{(1)}_+E^{(2)}_-;\\[4pt]
\underline{\sigma^+\sigma^+:}\quad
&g_{+}^{(+)}=\big[\chi^{(+)}_{(+1),A}+\chi^{(+)}_{(+1),B}+\chi^{(+)}_{(+1),C}\big]E^{(1)}_+E^{(2)}_+,
&&\kappa_{+}^{(+)}=\big[\chi^{(+)}_{(-1),A}+\chi^{(+)}_{(-1),B}+\chi^{(+)}_{(-1),C}\big]E^{(1)}_+E^{(2)}_+,\\
&g_{-}^{(+)}=\big[\chi^{(+)}_{(-1),A}+\chi^{(+)}_{(-1),B}+\chi^{(+)}_{(-1),C}\big]E^{(1)}_+E^{(2)}_+,
&&\kappa_{-}^{(+)}=\big[\chi^{(+)}_{(+1),A}+\chi^{(+)}_{(+1),B}+\chi^{(+)}_{(+1),C}\big]E^{(1)}_+E^{(2)}_+,\\
&g_{+}^{(-)}=\big[\chi^{(-)}_{(+1),A}-(\chi^{(-)}_{(+1),B}+\chi^{(-)}_{(+1),C})\big]E^{(1)}_+E^{(2)}_+,
&&\kappa_{+}^{(-)}=\big[\chi^{(-)}_{(-1),A}-(\chi^{(-)}_{(-1),B}+\chi^{(-)}_{(-1),C})\big]E^{(1)}_+E^{(2)}_+,\\
&g_{-}^{(-)}=\big[\chi^{(-)}_{(-1),A}-(\chi^{(-)}_{(-1),B}+\chi^{(-)}_{(-1),C})\big]E^{(1)}_+E^{(2)}_+,
&&\kappa_{-}^{(-)}=\big[\chi^{(-)}_{(+1),A}-(\chi^{(-)}_{(+1),B}+\chi^{(-)}_{(+1),C})\big]E^{(1)}_+E^{(2)}_+.
\end{align*}

\subsection{Solution and time-resolved Faraday rotation}

For steady input $s_{\mathrm{in},\alpha}(t)=s_{\mathrm{in},\alpha}$ and weak mixing, set $\dot a_{\pm,\alpha}\!\approx\!0$:
\[
a_{\pm,\alpha}\approx \frac{i\,g_{\pm}^{(\alpha)}}{\gamma_{\pm}+i\delta_{\pm}}\,a_{0,\alpha},\qquad
\delta_{\pm}=\delta_0\pm\Delta.
\]
The carrier then obeys
\[
\dot a_{0,\alpha}=\Big(-\gamma_{0,\alpha}-i\delta_{0,\alpha}\Big)a_{0,\alpha}
+i\Sigma_{\alpha}\,a_{0,\alpha}
+\sqrt{2\gamma_{e,\alpha}}\,s_{\mathrm{in},\alpha},\qquad
\Sigma_{\alpha}\equiv
\frac{\kappa_{+}^{(\alpha)}g_{+}^{(\alpha)}}{\gamma_{+}+i\delta_{+}}
+\frac{\kappa_{-}^{(\alpha)}g_{-}^{(\alpha)}}{\gamma_{-}+i\delta_{-}}.
\]
Steady state:
\[
a_{0,\alpha}=\frac{\sqrt{2\gamma_{e,\alpha}}\,s_{\mathrm{in},\alpha}}
{\gamma_{0,\alpha}+i\delta_{0,\alpha}-i\Sigma_{\alpha}}.
\]
The detected probe (carrier+sidebands) is
\[
\mathbf E(t)\approx 
\sum_{\alpha=\pm}\Big[
a_{0,\alpha}\,e^{-i\omega_s t}
+a_{+,\alpha}\,e^{-i(\omega_s+\Delta)t}
+a_{-,\alpha}\,e^{-i(\omega_s-\Delta)t}
\Big]\hat{\mathbf e}_{\alpha}.
\]
Define small sideband ratios $r_{\pm}^{(\alpha)}\equiv a_{\pm,\alpha}/a_{0,\alpha}=i\,g_{\pm}^{(\alpha)}/(\gamma_{\pm}+i\delta_{\pm})$.
To first order in $r_{\pm}^{(\alpha)}$, the instantaneous polarization angle (Faraday rotation) is
\begin{equation}
\boxed{~
\theta(t)
=\underbrace{\frac{k_0 L}{4}\,\mathrm{Re}\big[\chi^{\mathrm{eff,SB}}_{++}(\omega_s)-\chi^{\mathrm{eff,SB}}_{--}(\omega_s)\big]}_{\displaystyle \theta_0\ \text{(static, carrier)}}
\ +\ \underbrace{\frac{1}{2}\,\big|R\big|\cos\!\big(\Delta t+\arg R\big)}_{\text{oscillation at }\Delta},
\qquad
R\ \equiv\ r_{+}^{(+)}+r_{-}^{(+)}-\big(r_{+}^{(-)}+r_{-}^{(-)}\big).
~}
\label{eq:theta_time}
\end{equation}
The static piece $\theta_0$ reflects circular birefringence of the \emph{carrier} alone; the oscillatory part arises
from carrier–sideband interference and vanishes if the sidebands are spectrally filtered out.
For $\sigma^+\sigma^-$ with equal intensities in an isotropic medium, the stationary carrier contribution often
satisfies $\chi^{\mathrm{eff}}_{++}=\chi^{\mathrm{eff}}_{--}$, so $\theta_0\!\approx\!0$ and rotation is dominantly the $\Delta$-oscillation.
For $\sigma^+\sigma^+$, $\theta_0\neq0$ already without sidebands, which then dress it and add the $\Delta$ term.



% ---------------------------------------------------------------
% Subsection: TCMT at the carrier for circular polarizations
% (captures Optica 2019 within our notation)
% ---------------------------------------------------------------
\subsection{TCMT at the carrier for circular polarizations (cavity form)}

We now specialize to a resonant geometry (single cavity resonance at $\omega_0$) and
write a temporal coupled–mode theory (TCMT) \cite{FanSuhJoannopoulos2003}
for the \emph{carrier-frequency} probe envelopes in the circular basis.
Let $a_{0,+}(t)$ and $a_{0,-}(t)$ denote the intracavity envelopes of the probe’s
$\sigma^\pm$ components at the carrier $\omega_s\simeq \omega_0$; the external
drives (ports) are $s_{\rm in,\pm}(t)$ with the standard input–output relation
$s_{\rm out,\pm}=s_{\rm in,\pm}-\sqrt{2\gamma_e}\,a_{0,\pm}$.
Two strong pumps at $\omega_{p\pm}=\omega_p\pm\Delta/2$ and opposite helicities
($\sigma^+$, $\sigma^-$) bias the cavity and produce a time-periodic mixing at $\Delta$
between the circular probe modes (the engine behind the \emph{optically driven Faraday effect} \cite{DugganOptica2019}).

\paragraph{Time-periodic TCMT (lab frame).}
Under the instantaneous Kerr assumption and after projecting the $\chi^{(3)}$
tensor onto the circular basis (our isotropic $A,B,C$ reduction),
the \emph{carrier} envelopes obey
\begin{equation}
\boxed{~
\begin{aligned}
\dot a_{0,+} &= \big(-\gamma_0 - i\delta_0\big)\,a_{0,+}
\;+\; i\,D\,e^{+i\Delta t}\,a_{0,-}
\;+\;\sqrt{2\gamma_e}\,s_{\rm in,+}(t),\\[2pt]
\dot a_{0,-} &= \big(-\gamma_0 - i\delta_0\big)\,a_{0,-}
\;+\; i\,D^{*}e^{-i\Delta t}\,a_{0,+}
\;+\;\sqrt{2\gamma_e}\,s_{\rm in,-}(t),
\end{aligned} ~}
\label{eq:cavity-tcmt-lab}
\end{equation}
where $\gamma_0$ is the intrinsic decay (half-linewidth), $\delta_0\equiv \omega_s-\omega_0$
is the probe–cavity detuning, and $D$ is a pump-controlled complex coupling
proportional to the appropriate $\chi^{(3)}$ Floquet block(s) contracted with the pump Jones vectors
(e.g., for $\sigma^+\sigma^-$ pumps $D\propto \big[\chi^{(+)}_{(+1),B}+\chi^{(+)}_{(+1),C}\big]E^{(1)}_+E^{(2)\!*}_-$).
Equation~\eqref{eq:cavity-tcmt-lab} is the TCMT analogue of the time-periodic polarization
coupling used in Ref.~\cite{DugganOptica2019} (their $e^{\pm i\Delta t}$ mixing).

\paragraph{Static coupling via a rotating frame.}
Define a slow rotating frame $b_\pm(t)\equiv a_{0,\pm}(t)\,e^{\mp i\Delta t/2}$.
Then Eqs.~\eqref{eq:cavity-tcmt-lab} become
\begin{equation}
\boxed{~
\begin{aligned}
\dot b_{+} &= \Big(-\gamma_0 - i(\delta_0 - \tfrac{\Delta}{2})\Big)\,b_{+}
\;+\; i\,D\,b_{-}
\;+\;\sqrt{2\gamma_e}\,s_{\rm in,+}(t)\,e^{-i\Delta t/2},\\[2pt]
\dot b_{-} &= \Big(-\gamma_0 - i(\delta_0 + \tfrac{\Delta}{2})\Big)\,b_{-}
\;+\; i\,D^{*}\,b_{+}
\;+\;\sqrt{2\gamma_e}\,s_{\rm in,-}(t)\,e^{+i\Delta t/2}.
\end{aligned} ~}
\label{eq:cavity-tcmt-rot}
\end{equation}
For quasi-CW probing near the cavity line, the exponential factors on the drives
are slowly varying constants, so the coupling becomes \emph{static}.
This $2\times 2$ system is convenient for closed-form steady-state solutions and
matches the effective-Faraday cavity treatment of \cite{DugganOptica2019}.

\paragraph{Closed-form steady state and rotation.}
With $\dot b_\pm=0$ and constant inputs, write
\begin{equation}
\begin{bmatrix} b_+ \\[2pt] b_- \end{bmatrix}
=
\underbrace{\begin{bmatrix}
\gamma_0+i(\delta_0-\tfrac{\Delta}{2}) & -\,iD \\
-\,iD^{*} & \gamma_0+i(\delta_0+\tfrac{\Delta}{2})
\end{bmatrix}^{-1}}_{\displaystyle \mathbf G(\delta_0,\Delta,D)}
\begin{bmatrix}
\sqrt{2\gamma_e}\,s_{\rm in,+} \\[2pt]
\sqrt{2\gamma_e}\,s_{\rm in,-}
\end{bmatrix}.
\label{eq:cavity-green}
\end{equation}
Transform back to the lab frame via $a_{0,\pm}=b_\pm e^{\pm i\Delta t/2}$ and use
$s_{\rm out,\pm}=s_{\rm in,\pm}-\sqrt{2\gamma_e}\,a_{0,\pm}$ to obtain the port fields.
The \emph{instantaneous} polarization rotation is
\begin{equation}
\boxed{~
\theta(t)=\tfrac{1}{2}\,\big[\arg a_{0,+}(t)-\arg a_{0,-}(t)\big].
~}
\label{eq:theta-cavity}
\end{equation}
Near a cavity line, the observable rotation is essentially \emph{static} (the
$e^{\pm i\Delta t/2}$ factors cancel in the Stokes parameters over the cavity lifetime),
and one may define
\begin{equation}
\boxed{~
\theta_0=\tfrac{1}{2}\arg\!\left(\frac{b_+}{b_-}\right)
=\tfrac{1}{2}\arg\!\left(
\frac{\gamma_0+i(\delta_0-\tfrac{\Delta}{2})}{\gamma_0+i(\delta_0+\tfrac{\Delta}{2})}
\;+\;
\frac{|D|^2}{\big[\gamma_0+i(\delta_0+\tfrac{\Delta}{2})\big]\big[\gamma_0+i(\delta_0-\tfrac{\Delta}{2})\big]}
\right).
~}
\label{eq:theta0-cavity}
\end{equation}
Equations~\eqref{eq:cavity-tcmt-lab}–\eqref{eq:theta0-cavity} are the cavity counterpart of our
sideband-elimination picture: the Lorentzian dispersive factor $1/(\Gamma_s\mp i\Delta)$ appearing in the
bulk propagation is here replaced by the cavity susceptibilities
$1/(\gamma_0+i(\delta_0\mp \Delta/2))$, while the mixing amplitude $D$ contains the
same $\chi^{(3)}$ contractions with the pump helicities as in our isotropic $A,B,C$ expansion.
This reproduces the \emph{optically driven effective Faraday effect} from Ref.~\cite{DugganOptica2019}
within our notation, and fits naturally in the TCMT framework \cite{FanSuhJoannopoulos2003}.

Of course, there can be typos and stupid mistakes, but the whole logic is correct.


\section{References}
\begin{thebibliography}{9}
\bibitem{Boyd}
R.~W.~Boyd, \textit{Nonlinear Optics} (4th ed.), Academic Press (2020). See frequency-domain polarization and the SI prefactor linking $n_2$ and $\mathrm{Re}\,\chi^{(3)}$. URL: \url{https://lib.ysu.am/disciplines_bk/cce1226f312d5f73b4d8dd86e878c5b3.pdf}.

\bibitem{SheikBahae}
M.~Sheik-Bahae, ``Third-Order Optical Nonlinearities,'' in \textit{Handbook of Optics, Vol.~IV} (OSA/McGraw-Hill). A convenient overview of frequency selection rules and generated frequencies. URL: \url{https://msbahae.unm.edu/Courses/568%20Nonlinear%20Optics/OSA-Handbook%20of%20Optics-IV-Ch17.pdf}.

\bibitem{CREOL}
CREOL Lecture Notes, ``Third-order nonlinear susceptibility $\chi^{(3)}$ and its tensor representation,'' discussing isotropic tensor reduction and Kleinman symmetry. URL: \url{https://www.creol.ucf.edu/mir/wp-content/uploads/sites/7/2023/07/L16_-3rd-order-nonlinear-susceptibility-%CF%873-and-its-tensor-representation.pdf}.

\bibitem{Stokes}
Stokes parameters and $S_3$ definition (circular component). URL: \url{https://en.wikipedia.org/wiki/Stokes_parameters}.

\bibitem{UTAustin}
UT Austin Lecture Notes, Faraday rotation and circular birefringence relation $\theta=(k_0L/2)(n_+-n_-)$. URL: \url{https://farside.ph.utexas.edu/teaching/jk1/Electromagnetism/node73.html}.

\bibitem{DugganOptica2019}
R.~Duggan, D.~Sounas, and A.~Al\`u,
``Optically driven effective Faraday effect in instantaneous nonlinear media,''
\textit{Optica} \textbf{6}(9), 1152--1157 (2019).
doi:10.1364/OPTICA.6.001152

\bibitem{FanSuhJoannopoulos2003}
S.~Fan, W.~Suh, and J.~D.~Joannopoulos,
``Temporal coupled-mode theory for the Fano resonance in optical resonators,''
\textit{Journal of the Optical Society of America A} \textbf{20}(3), 569--572 (2003).
doi:10.1364/JOSAA.20.000569

\end{thebibliography}

\end{document}
